\documentclass[17pt, margin=1in, innermargin=-2.5in, blockverticalspace=-0.27in]{tikzposter}
\geometry{paperwidth=33.11in,paperheight=46.81in} %A0

\linespread{1.05}
\setlength{\parskip}{0.3em}

% \geometry{paperheight=33.11in,paperwidth=23.4in} %A1
\usepackage[utf8]{inputenc}
\usepackage[T2A]{fontenc}
\usepackage[russian]{babel}
\usepackage{csquotes}
\usepackage{amsmath}
\usepackage{amsfonts}
\usepackage{amsthm}
\usepackage{amssymb}
\usepackage{mathrsfs}
\usepackage{graphicx}
\usepackage{lipsum}
\usepackage[export]{adjustbox}
\usepackage{enumitem}
\usepackage[backend=biber, style=ieee]{biblatex}

\usepackage{xcolor}
\usepackage[font=normal, labelfont=bf, width=1.3\linewidth]{caption}

\usepackage[skip=0cm, list=true, labelfont=bf]{subcaption}
%\usepackage{floatrow}
\usepackage{fouriernc}
\usepackage{tempora}
\usepackage{mathtools}
%\usepackage[unicode=true,hidelinks]{hyperref}
%\usepackage[fixlanguage]{babelbib}
\usepackage[oldsyntax]{stackengine}
\usepackage{scalerel}
\usepackage{euscript}

\setlength{\abovecaptionskip}{6pt plus 2pt minus 2pt}
\setlength{\belowcaptionskip}{16pt plus 4pt minus 4pt}

\renewcommand{\figurename}{Fig}


% University of Tartu Computer Science Theme for the tikzposter
% package.
% Author: Joonas Puura, joonas.puura@ut.ee

% Adapted from University of Bristol theme by 
        % Author: Nicholas P Baskerivlle
        % adapted by Rene Welch
        
% Last Modified: 2021-01-10

% Colours are based on style book for University of Tartu: http://stiiliraamat.ut.ee/
% -- COLOURS --


% Primary colour UT
\definecolor{MainBlue}{HTML}{2C5696} % Tunnusvärv, Sinine, Pantone 7685
\definecolor{LightBlue}{HTML}{00A6E6} % Põhipaleti täiendvärv, Helesinine, Pantone 2995
\definecolor{DarkBlue}{HTML}{000000} % Põhipaleti täiendvärv, Tumesinine, SPOT 281 
\definecolor{Orange}{HTML}{c35413} % Aktsentvärv, Oranž, Pantone 151

% Additional colors UT
\definecolor{Gray}{HTML}{B1B3B3} % Lisavärv, Hall, Pantone Cool Gray 5

% Blue as percentage
\colorlet{Blue10}{MainBlue!10} % 10%
\colorlet{Blue20}{MainBlue!20} % 20%
\colorlet{Blue30}{MainBlue!30} % 30%
\colorlet{Blue40}{MainBlue!40} % 40%

% Faculty colours
\definecolor{MedPink}{HTML}{E52143}% Medicine, Pantone 192
\definecolor{AHViolet}{HTML}{AE78B1} % Arts and Humanities, Pantone 7440
\definecolor{STGreen}{HTML}{87BC1F} % Science and Technology, Pantone 376
\definecolor{SocOrange}{HTML}{FAA41A} % Social Sciences, Pantone 137

% Other colours
\definecolor{Black}{HTML}{101820} % Pantone Black 6


% tikzposter color palette

\definecolorpalette{UniTartuPalette} {
    \definecolor{colorOne}{named}{white}
    \definecolor{colorTwo}{named}{Black}
    \definecolor{colorThree}{named}{Black}
}

% tikzposter style
\definecolorstyle{UniTartuStyle} {
    \usecolorpalette{UniTartuPalette}
}{
    % background
    \colorlet{backgroundcolor}{white}
    \colorlet{framecolor}{white}
    % title colors
    \colorlet{titlefgcolor}{Black}
    \colorlet{titlebgcolor}{white}
    % block colors
    \colorlet{blocktitlebgcolor}{colorOne}
    \colorlet{blocktitlefgcolor}{Black}
    \colorlet{blockbodybgcolor}{white}
    \colorlet{blockbodyfgcolor}{Black}
    % innerblock colors
    \colorlet{innerblocktitlebgcolor}{Black}
    \colorlet{innerblocktitlefgcolor}{Black}
    \colorlet{innerblockbodybgcolor}{colorTwo}
    \colorlet{innerblockbodyfgcolor}{Black}
    % note colors
    \colorlet{notefgcolor}{Black}
    \colorlet{notebgcolor}{colorTwo}
    \colorlet{noteframecolor}{colorTwo}
}

% -- STYLE --

% background
\definebackgroundstyle{UniTartuBackgroundStyle}{
    \draw[line width=0pt, color=framecolor, fill=backgroundcolor]
    (bottomleft) rectangle (topright);
}

% title
\definetitlestyle{UniTartuTitleStyle}{
    width=\textwidth, linewidth=2pt, titletotopverticalspace=0in
}{
    \begin{scope}[line width=\titlelinewidth,]
    \draw[color=colorThree,round cap-round cap]
    (\titleposleft,\titleposbottom)--(\titleposright,\titleposbottom);
    \end{scope}
}

% block
\defineblockstyle{UniTartuBlockStyle}{
    titlewidthscale=1.0, bodywidthscale=1.0, roundedcorners=5
}{
    \draw[color=framecolor, fill=blockbodybgcolor,
    rounded corners=\blockroundedcorners] (blockbody.south west)
    rectangle (blockbody.north east);
    \ifBlockHasTitle
    \draw[color=framecolor, fill=blocktitlebgcolor,
    rounded corners=\blockroundedcorners] (blocktitle.south west)
    rectangle (blocktitle.north east);
    \fi
}

% -- THEME -- 
\definelayouttheme{UniTartuTheme}{
    \usecolorstyle[colorPalette=UniTartuPalette]{UniTartuStyle}
    \usebackgroundstyle{UniTartuBackgroundStyle}
    \usetitlestyle{UniTartuTitleStyle}
    \useblockstyle{UniTartuBlockStyle}
    \useinnerblockstyle{Default}
    \usenotestyle{Default}
}

% -- TITLE FORMAT --

% place logo to right of centered title
\makeatletter
\renewcommand\TP@maketitle{%
%   \centering
   \begin{minipage}[b]{0.95\linewidth}
        % \centering
        \color{titlefgcolor}
        {\bfseries \Huge \sc \@title \par}
        \vspace*{1em}
        {\huge \@author \par}
        \vspace*{1em}
        {\LARGE \@institute}
    \end{minipage}%
    \hfill
    \tikz[remember picture,overlay]\node[anchor=south east,xshift=0.42\linewidth,yshift=0.5cm,inner sep=0pt] {%
        \@titlegraphic
    };
}
\makeatother

%%%%%%%%%%%%%%%%%%%%%%%%%%%%%%%%%%%

\AtBeginEnvironment{tikzfigure}{\captionsetup{type=figure}}

%% ------------------------- %

\newcommand{\shifthat}[2]{%
    \stackengine{\Sstackgap}{$#2$}{\(\hspace{#1}\hat{}\)}{O}{l}{F}{T}{S}
}

% ------------------------- %

\newcommand{\operator}[2][operator]{
    \if H#2\shifthat{0.5em}{#2}\else
    \if d#2\shifthat{0.49em}{#2}\else
    \if q#2\shifthat{0.35em}{#2}\else
    \if \mu#2\shifthat{0.35em}{#2}\else
    \shifthat{0.45em}{#2}
    \fi
    \fi
    \fi
    \fi
}

% ------------------------- %

\newcommand{\vectoperator}[2][operator]{
    \if d#2\shifthat{0.367em}{\textbf{#2}}\else
    \if m#2\shifthat{0.4em}{\textbf{#2}}\else
    \shifthat{0.275em}{\textbf{#2}}
    \fi
    \fi
}

% ------------------------- %

\newcommand{\vect}[4][vector]{
    {#2_{#3}^{#4}}
    %\overrightarrow{#2_{#3}} % with arrow
}

% ------------------------- %

\newcommand{\vectbf}[2][bold vector]{
    \vect{\boldsymbol{#2}}
}

% ------------------------- %

\newcommand{\pd}[3][empty]{
    \frac{\partial{#2}}{\partial{#3}}
}

% ------------------------- %

\newcommand{\func}[5][empty]{
    {#2}_{#3}^{#4} \left({#5} \right)
}

% ------------------------- %

\newcommand{\funccomp}[5][empty]{
    {#2}_{#3}^{#4} ({#5})
}

% ------------------------- %

\newcommand{\underrel}[3][]{
    \mathrel{\mathop{#3}\limits_{
        \if x c#1\relax\mathclap{#2}\else#2\fi
    }}
}

% ------------------------- %

\newcommand{\ifrac}[2]{
    {#1}\:/\:{#2}
}
%% ------------------------- %

\makeatletter

\newcommand{\Autoref}[1]{\@first@ref#1,@}
\def\@throw@dot#1.#2@{#1}% discard everything after the dot
\def\@set@refname#1{%    % set \@refname to autoefname+s using \getrefbykeydefault
    \edef\@tmp{\getrefbykeydefault{#1}{anchor}{}}%
    \xdef\@tmp{\expandafter\@throw@dot\@tmp.@}%
    \ltx@IfUndefined{\@tmp autorefnameplural}%
        {\def\@refname{\@nameuse{\@tmp autorefname}}}%
        {\def\@refname{\@nameuse{\@tmp autorefnameplural}}}%
}
\def\@first@ref#1,#2{%
    \ifx#2@\autoref{#1}\let\@nextref\@gobble% only one ref, revert to normal \autoref
    \else%
        \@set@refname{#1}%  set \@refname to autoref name
        \@refname~\ref{#1}% add autoefname and first reference
        \let\@nextref\@next@ref% push processing to \@next@ref
    \fi%
    \@nextref#2%
}
\def\@next@ref#1,#2{%
    \ifx#2@ и~\ref{#1}\let\@nextref\@gobble% at end: print and+\ref and stop
    \else, \ref{#1}% print  ,+\ref and continue
    \fi%
    \@nextref#2%
}

\makeatother

% ------------------------- %
%% ------------------------- %

\newcommand{\img}[4][anything]{
    \begin{figure}[H]{
        \center{\includegraphics[width={#4}]{{#1}}}
        \caption{#2}\label{#3}}
    \end{figure}
}

% ------------------------- %

\newcommand{\floatimg}[4][anything]{
    \begin{figure}[ht]{
        \center{\includegraphics[width={#4}]{{#1}}}
        \caption{#2}\label{#3}}
    \end{figure}
}

% ------------------------- %

\newcommand{\subimg}[2][anything]{
    \begin{minipage}[h]{{#2}} % 0.4\textwidth
        \center{\includegraphics[width=1\linewidth]{{#1}}}
    \end{minipage}
}

% ------------------------- %

\newcommand{\subimgtwo}[4][anything]{
    \subfloat[{#2}]{\includegraphics[width={#4}]{{#1}}\label{#3}}
}

% ------------------------- %
%\newsavebox{\foobox}

\newcommand{\slantbox}[2][0]{\colorlet{slantcolor}{.}\mbox{%
        \sbox{\foobox}{\color{slantcolor}#2}%
        \hskip\wd\foobox
        \pdfsave
        \pdfsetmatrix{1 0 #1 1}%
        \llap{\usebox{\foobox}}%
        \pdfrestore
}}
\newcommand\unslant[2][-.2]{%
  \mkern1mu%
  \ThisStyle{\slantbox[#1]{$\SavedStyle#2$}}%
  \mkern-1mu%
}
\newcommand\upmu{\unslant\mu}

%%%%%%%%%%%%%%%%%%%%%%%%%%%%%%%%%%%

\makeatletter
\setlength{\TP@visibletextwidth}{31.0in}
\setlength{ \TP@visibletextheight}{45in}
\makeatother
\usepackage{mwe} % for placeholder images
\usepackage{bm}
\usepackage{bbm}



% set theme parameters
\tikzposterlatexaffectionproofoff
\usetheme{UniTartuTheme}
\usecolorstyle{UniTartuStyle}
% ------------------------- %

\newcommand{\shifthat}[2]{%
    \stackengine{\Sstackgap}{$#2$}{\(\hspace{#1}\hat{}\)}{O}{l}{F}{T}{S}
}

% ------------------------- %

\newcommand{\operator}[2][operator]{
    \if H#2\shifthat{0.5em}{#2}\else
    \if d#2\shifthat{0.49em}{#2}\else
    \if q#2\shifthat{0.35em}{#2}\else
    \if \mu#2\shifthat{0.35em}{#2}\else
    \shifthat{0.45em}{#2}
    \fi
    \fi
    \fi
    \fi
}

% ------------------------- %

\newcommand{\vectoperator}[2][operator]{
    \if d#2\shifthat{0.367em}{\textbf{#2}}\else
    \if m#2\shifthat{0.4em}{\textbf{#2}}\else
    \shifthat{0.275em}{\textbf{#2}}
    \fi
    \fi
}

% ------------------------- %

\newcommand{\vect}[4][vector]{
    {#2_{#3}^{#4}}
    %\overrightarrow{#2_{#3}} % with arrow
}

% ------------------------- %

\newcommand{\vectbf}[2][bold vector]{
    \vect{\boldsymbol{#2}}
}

% ------------------------- %

\newcommand{\pd}[3][empty]{
    \frac{\partial{#2}}{\partial{#3}}
}

% ------------------------- %

\newcommand{\func}[5][empty]{
    {#2}_{#3}^{#4} \left({#5} \right)
}

% ------------------------- %

\newcommand{\funccomp}[5][empty]{
    {#2}_{#3}^{#4} ({#5})
}

% ------------------------- %

\newcommand{\underrel}[3][]{
    \mathrel{\mathop{#3}\limits_{
        \if x c#1\relax\mathclap{#2}\else#2\fi
    }}
}

% ------------------------- %

\newcommand{\ifrac}[2]{
    {#1}\:/\:{#2}
}

%\usepackage[scaled]{helvet}
%\renewcommand\familydefault{\sfdefault} 
%\renewcommand{\vec}[1]{\bm{#1}}
%\newcommand{\Tr}{\text{Tr}}

\addbibresource{refs.bib}
\addbibresource{components/bibliography.bib}

\title{\parbox{0.8\linewidth}{Усиление угловой дисперсии лазерных гармоник высокого порядка при взаимодействии с плотными плазменными кластерами}}
\author{\textbf{Л.А. Литвинов}\textsuperscript{1}, \textbf{А.А. Андреев}\textsuperscript{1, 2}}
\institute{\textsuperscript{1}\textit{Санкт-Петербургский государственный университет, 199034 Санкт-Петербург, Россия} \\ \textsuperscript{2}\textit{Физико-технический институт имени А.Ф.Иоффе, 188640 Санкт-Петербург, Россия}}

%\titlegraphic{\includegraphics[width=0.06\linewidth]{../components/img/spbu_logo2.png}}

% begin document
\begin{document}
    \maketitle
    \centering
    \begin{columns}
        \column{0.33}

        \block{}{
            Периодические поверхностные решетки или фотонные кристаллы — отличные инструменты для управления светом. Однако этот метод менее эффективен в случае экстремального ультрафиолетового света из-за высокого поглощения любого материала в этом диапазоне частот. В работе исследуется возможность углового усиления такого излучения при помощи рассеяния на подходящих сферических кластерах. Была разработана аналитическая модель с использованием диэлектрической функции плазмы Друде и теории рассеяния Ми. Модель построена в квазистатическом приближении, так как время ионизации меньше длительности импульса, что значительно меньше времени разлета плазмы. Оценены резонансные параметры мишени по десятой гармонике титан-сапфирового лазера и найдено усиление рассеянного поля в резонансном случае по сравнению с первой гармоникой. Используя те же условия резонанса для одного кластера, мы моделируем дифракцию на массиве таких кластеров с помощью кода CELES. Полученные результаты показывают значительное усиление рассеянного поля в резонансном случае для больших углов, что соответствует теории дифракции Брэгга-Вульфа, --- возможность управления высокими гармониками лазерного излучения в XUV-диапазоне с помощью ионизированного кластерного газа.
        }

        \block[bodyoffsety=0.7cm]{Введение}{
            \setlength{\parindent}{1em}
            В пределах микрометровых длин волн фотонные кристаллы и решетки могут использоваться для направления или дифрагирования электромагнитных волн, в то время как аналогичные рентгеновские манипуляции могут использовать кристаллы с атомами, регулярно расположенными на расстоянии нескольких нанометров друг от друга, в качестве рассеивающих центров. В то же время большой разрыв между этими диапазонами длин волн, называемый XUV (экстремальным ультрафиолетом) или жестким ультрафиолетом, оказывается сложным для манипулирования. Для решения этой проблемы предлагается использовать массивы сферических нанокластеров для направленного рассеяния жесткого ультрафиолетового излучения (Рис.~\ref{intsch:image}).

\begin{tikzfigure}
    \includegraphics[width=0.75\linewidth]{../img/plasma_area2}\label{intsch:image}\caption{Схема взаимодействия. Плоскость поляризации параллельна одной из граней кубической области. Размеры сферических кластеров порядка нескольких нанометров, а расстояние между ними составляет не менее сотен нанометров. Распределение кластеров внутри кубической области в общем случае произвольно.}
\end{tikzfigure}
        }

        \block[bodyoffsety=1cm]{Аналитическая модель}{
            \setlength{\parindent}{1em}
            Рассмотрим одиночный кластер радиусом $a$, облучаемый коротким фемтосекундным импульсом с интенсивностью $I_{h} \approx 10^{14}$ $\textrm{Вт/см}^2$. Модель Друде дает представление о диэлектрической функции плазмы:

    \begin{equation}
		\varepsilon (\omega) = 1 - {\left( \frac{\omega_{pe}}{\omega} \right)}^2 \frac{1}{1+i \beta_{e}}, \qquad \omega_{pe} = \sqrt{\frac{4 \pi e^2 n_e}{m_e}},
		\label{eps_plasma}
    \end{equation}
    \begin{equation*} % artificial indent after the equation
    \end{equation*}

\noindent где $\omega$ --- частота рассматриваемой гармоники, $\omega_{pe}$ --- плазменная частота электронов, $n_e = Z n_i$ --- электронная плотность, $Z$ --- средняя степень ионизации, $n_i$ --- ионная плотность. $\beta_{e} = v_e / \omega$ и $v_e$ --- коэффициент электрон-ионных столкновений в приближении Спитцера.

\begin{tikzfigure}
  \includegraphics[width=0.7\linewidth]{../img/single_sph_scheme}\label{single_sph_scheme:image}\caption{Схема аналитической модели.}
\end{tikzfigure}

Теорию Ми можно использовать для описания рассеянного поля и поля внутри рассеивающего объекта. Рассмотрим сферический кластер и $x$-поляризованную плоскую волну, распространяющуюся вдоль оси $z$ (Рис.~\ref{single_sph_scheme:image}). Используя обобщенное разложение Фурье в случае изотропной среды, имеем следующий вид коэффициентов рассеянного поля~\cite{boren_huffman}:

    \begin{equation}
		\vectbf{E}{s} = \sum_{n = 1}^{\infty}E_n \left[ i a_n\left(ka, m\right) \vectbf{N}{}^{(3)}_{e1n} - b_n\left(ka, m\right) \vectbf{M}{}^{(3)}_{o1n} \right], \quad E_n = i^{n} E_0 \frac{2n + 1}{n \left(n + 1\right)},
        \label{E_s_sph}
    \end{equation}

    \begin{equation}
		a_n(x,\:m) = \frac{m \func{\psi}{n}{\prime}{x} \func{\psi}{n}{}{mx} - \func{\psi}{n}{\prime}{mx} \func{\psi}{n}{}{x}}{m \func{\xi}{n}{\prime}{x} \func{\psi}{n}{}{mx} - \func{\psi}{n}{\prime}{mx} \func{\xi}{n}{}{x}},
		\label{an_bessel}
    \end{equation}

    \begin{equation}
        b_n(x,\:m) = \frac{\func{\psi}{n}{\prime}{x} \func{\psi}{n}{}{mx} - m \func{\psi}{n}{\prime}{mx} \func{\psi}{n}{}{x}}{\func{\xi}{n}{\prime}{x} \func{\psi}{n}{}{mx} - m \func{\psi}{n}{\prime}{mx} \func{\xi}{n}{}{x}},
        \label{bn_bessel}
    \end{equation}
    \begin{equation*} % artificial indent after the equation
    \end{equation*}

\noindent где $\funccomp{\psi}{n}{}{\rho} = \rho \funccomp{j}{n}{}{\rho}$, $\funccomp{\xi}{n}{}{\rho} = \rho \funccomp{h}{n}{}{\rho}$ --- функции Риккати-Бесселя, $h_n = j_n + i \gamma_n$ --- сферические функции Ганкеля первого рода, $x = ka$ --- сферические функции Ганкеля первого рода, $ m = \sqrt{\varepsilon} $ --- комплексный показатель преломления.

% ============================= %

\begin{tikzfigure}
  \subcaptionbox{$ka = 0.5$, zero order approximation.}{
      \includegraphics[width=0.48\linewidth]{../img/sph_base/sph_ka0.5_123}
  }
  \hfil
  \subcaptionbox{$ka = 1.5$, first order approximation.}{
      \includegraphics[width=0.48\linewidth]{../img/sph_base/sph_ka1.5_123_1st}
  }
  \label{ab_asymp:image}\caption{Коэффициенты сферических гармоник в нулевом и первом приближении, $\beta_e = 0$. Кривые ``exact'' построены с использованием полных разложений.}
\end{tikzfigure}
        }

        \column{0.33}

        \block[bodyoffsety=0.7cm]{}{
            \setlength{\parindent}{1em}
            Используя аппроксимации функций Бесселя~\cite{boren_huffman} можно существенно упростить коэффициенты (\ref{an_bessel},~\ref{bn_bessel}) и получить условия резонанса для плотности электронов и радиуса кластера. 

        }

        \block[bodyoffsety=0.7cm]{Одиночный кластер}{
            \setlength{\parindent}{1em}
            Для проверки аналитической модели были рассчитано значение комплексного показателя преломления при $\lambda_{10} = 83$ нм, $ka = 0.7$. Мы рассматриваем случай, когда частота электрон-ионных столкновений $\nu_e$ много меньше частоты гармоники, поэтому взаимодействие можно считать бесстолкновительным~\cite{andreev_lecz}.

    \begin{tikzfigure}
        \subcaptionbox{$\lambda = \lambda_{L} = 830$ нм.}{
            \includegraphics[width=0.45\linewidth]{../img/mph_new/es_ka0.7_1harm}
        }
        \hfil
        \subcaptionbox{$\lambda = \lambda_{10} = 83$ нм.}{
            \includegraphics[width=0.45\linewidth]{../img/mph_new/es_ka0.7_10harm}
        }
        \label{ka0.7:image}\caption{$ka = 0.7$ ($a \approx 8.9$ нм); $|\vectbf{E}{s}|^2$ в плоскости поляризации падающей волны.}
    \end{tikzfigure}

В случае резонанса (Рис. 4а) амплитуда рассеянного поля вблизи кластера значительно выше, чем в отсутствие резонанса (Рис. 4б), когда рассеянные волны практически не наблюдаются.
        }

        \block[bodyoffsety=0.7cm]{Оправдание стационарной модели}{
            \setlength{\parindent}{1em}
            В общем случае расчет взаимодействия мощного лазерного импульса с группой плотных сферических кластеров, находящихся в трехмерном пространстве, требует длительных и сложных нестационарных расчетов из-за изменения электронной плотности кластеров во времени. Чтобы проверить масштаб такого изменения, мы смоделировали эволюцию распределения электронной плотности для одиночного одномерного кластера с помощью LPIC++~\cite{Pfund1998}.

\begin{tikzfigure}
    \includegraphics[width=0.6\linewidth]{../img/lpic/htr_over_2a_a}\label{lpic_htr:image}\caption{Асимптотическое поведение средней общей толщины переходного слоя при $0 \leq t \leq 10T$ по отношению к радиусу мишени. $n_c$ используемое в построении, соответствует критической плотности для длины волны $\lambda = \lambda_{10}$.}
\end{tikzfigure}
        }

        \block[bodyoffsety=1.4cm, titleoffsety=1.2cm]{Геометрия взаимодействия пучка с кластерами}{
            \setlength{\parindent}{1em}
            Условие дифракции в случае трехмерной регулярной решетки с упругим рассеянием~\cite{Kittel86} можно преобразовать следующим образом:

    \begin{equation}
        \begin{cases}
            \cos{\theta_0}\sin{\Delta \theta}\cos{\left( \Delta \varphi - \varphi_0 \right)} - \sin{\theta_0} \left( \cos{\Delta \theta} - 1 \right) = \cfrac{h^{\prime} \lambda}{d}
            \\
            \sin{\Delta \theta} \sin{\left( \Delta \varphi - \varphi_0 \right)} = \cfrac{k^{\prime} \lambda}{d}
            \\
            \sin{\theta_0}\sin{\Delta \theta}\cos{\left( \Delta \varphi - \varphi_0 \right)} + \cos{\theta_0} \left( \cos{\Delta \theta} - 1 \right)= \cfrac{l^{\prime} \lambda}{d}
        \end{cases}
        \label{bragg_wolf_order_spherical}
    \end{equation}
    \begin{equation*}
    \end{equation*}

\noindent где $\Delta \theta,\:\Delta \varphi$ --- углы, характеризующие отклонение направления дифрагированного излучения относительно падающего, $\theta_0,\:\varphi_0$ --- углы, характеризующие поворот мишени (решётки) в пространстве, $h^\prime,\:k^\prime,\:l^\prime$ --- индексы Миллера (Рис.~\ref{3ddiffr:image}), $\vectbf{e}{\textrm{in}} = \vectbf{e}{z}$ --- фиксированное направление падающего излучения, $d$ --- расстояние между центрами кластеров. Используя (\ref{bragg_wolf_order_spherical}), можно получить угловое распределение дифрагированного излучения при заданных начальных параметрах $d$, $\lambda$, $\theta_0$, $\varphi_0$.

\begin{tikzfigure}
    \subcaptionbox{Проекция на плоскость $xz$.}{
        \includegraphics[width=0.47\linewidth]{../img/article1_shortened_pic1}
    }
    \hfil
    \subcaptionbox{Проекция на плоскость $xy$.}{
        \includegraphics[width=0.47\linewidth]{../img/article1_shortened_pic2}
    }
    \label{3ddiffr:image}\caption{Общая схема взаимодействия падающего излучения с решеткой. $r_{\textrm{gas }}$ --- радиус газовой струи, представляющей мишень, $w$ --- ширина Гауссова пучка падающего излучения.}
\end{tikzfigure}
        }

        \block[bodyoffsety=1.4cm, titleoffsety=0.7cm]{Рассеяние монохроматического излучения}{
            \setlength{\parindent}{1em}
            В рамках стационарной теории рассеяния Ми многие кластеры рассматривались в виде протяженной цилиндрической газовой струи с регулярной и квазирегулярной пространственной конфигурацией. Квазирегулярное распределение строилось путем введения случайных сдвигов координат узлов с произвольной нормой сдвига в диапазоне $0 \leq |\Delta d| \leq \eta d$, где $0 \leq \eta < 0.5$ --- степень нерегулярности. Для расчетов использовался программный код CELES~\cite{celes}.

\begin{equation}
    E_{\textrm{int}} \left( \eta,\:\lambda, \:V\left(\:\Delta \theta,\:\Delta \varphi \right), \:E_0,\:\varphi_0,\:\theta_0 \right) = \int\limits_{V\left(\:\Delta \theta,\:\Delta \varphi \right)}  |\vectbf{E}{s}\left(\eta,\:\lambda,\:E_0,\:\varphi_0,\:\theta_0\right)|^2 dV,
    \label{e_int}
\end{equation}
\begin{align}
    \vectbf{c}{} = \vectbf{c}{}\left(\vectbf{x}{},\:\Delta \theta,\:\Delta \varphi \right) = M_y(\Delta \theta)\,M_z(\Delta \varphi)\:\vectbf{x}{}, \quad \vectbf{x}{} = \begin{pmatrix}x & y & z\end{pmatrix}^T,
    \label{c_for_e_int}
\end{align}
\begin{align}
    V\left(\:\Delta \theta,\:\Delta \varphi \right) = \left\{\vectbf{x}{} : c_{x}^2 + c_{y}^2 \leq \rho^2, \:\: b_1^2 \leq |\vectbf{x}{}|^2 \leq b_2^2 \right\}.
    \label{V_for_e_int}
\end{align}
        }

        \column{0.33}

        \block{}{
            \setlength{\parindent}{1em}
            \begin{tikzfigure}
    \subcaptionbox{$\lambda = \lambda_{L} = 830$ нм.}{
        \includegraphics[width=0.45\linewidth]{../img/celes/Es_20nm_15deg_1harm.pdf}
    }
    \hfil
    \subcaptionbox{$\lambda = \lambda_{10} = 83$ нм.}{
        \includegraphics[width=0.45\linewidth]{../img/celes/Es_20nm_15deg_10harm.pdf}
    }
    \\
    \subcaptionbox{Решение (\ref{bragg_wolf_order_spherical}) в целых индексах Миллера.}{
        \includegraphics[width=0.38\linewidth]{../img/celes/dphi_dtheta_kprime_d_2l_phi0_0_theta0_15.pdf}
    }
    \hfil
    \subcaptionbox{$E_{\textrm{int}}$ из (\ref{e_int}).}{
        \includegraphics[width=0.48\linewidth]{../img/celes/E_squared/eint_10harm_15deg_0.0nonreg.pdf}
    }
    \label{1st_check_diffrth:image}\caption{Рассеяние 10-ой гармоники при $a = 20$ нм и $d = 2\lambda_{10}$, $\varphi_0 = 0^{\circ}$, $\theta_0 = 15^{\circ}$, $\lambda = \lambda_{10} = 83$ nm, $\Delta \theta \in \left[ 0,\:\pi\,/\,2 \right]$.}
\end{tikzfigure}

\begin{tikzfigure}
    \subcaptionbox{Влияние нерегулярности мишени.}{
        \includegraphics[width=0.48\linewidth]{../img/celes/energy_vs_nonreg}
    }
    \hfil
    \subcaptionbox{Влияние радиуса кластеров, значения нормированы на $E_{\textrm{int}}$ при $a = 20$ нм.}{
        \includegraphics[width=0.48\linewidth]{../img/celes/energy_vs_radius}
    }
    \label{scat_at:image}\caption{Затухание рассеяния в зависимости от нерегулярности мишени и радиуса кластеров.}
\end{tikzfigure}
        }

        \block[bodyoffsety=2.1cm, titleoffsety=1.7cm]{Рассеяние волнового пакета}{
            \setlength{\parindent}{1em}
            \begin{tikzfigure}
    \subcaptionbox{Рассеяние 10-ой гармоники.}{
        \includegraphics[width=0.48\linewidth]{../img/celes/E_squared/eint_10harm_15deg_0.0nonreg.pdf}
    }
    \hfil
    \subcaptionbox{Рассеяние волнового пакета.}{
        \includegraphics[width=0.48\linewidth]{../img/celes/E_squared/eint_wavepacket2_15deg_0.0nonreg.pdf}
    }
    \label{wavepacket1:image}\caption{Диаграмма углового рассеяния гауссового волнового пакета и 10-ой гармоники. $\theta_0 = 15^\circ$, $\varphi_0 = 0^\circ$, $d = 2\lambda_{10}$, радиус кластеров $a = 20$ нм.}
\end{tikzfigure}

\begin{tikzfigure}
    \subcaptionbox{Рассеяние однорядным массивом кластеров.}{
        \includegraphics[width=0.48\linewidth]{../img/celes/plane_flat_to_compare.pdf}
    }
    \hfil
    \subcaptionbox{Рассеяние трёхрядным массивом кластеров.}{
        \includegraphics[width=0.48\linewidth]{../img/celes/3plane_flat_to_compare.pdf}
    }
    \label{cluster_rows:image}\caption{Рассеяние гармоники с $\lambda \approx 89$ нм массивом кластеров, $\varphi_0 = 0^\circ$, $\theta_0 = 30^\circ$, $a = 30$ нм, расстояние между центрами кластеров равно $d = 3\lambda$, падающее поле направлено от нижней от левого угла к правому верхнему под углом $\theta_0$ по отношению к нормали к мишени.}
\end{tikzfigure}
        }

        \block[bodyoffsety=2.1cm, titleoffsety=1.7cm]{Заключение}{
            \setlength{\parindent}{1em}
            \section{Заключение}

Было проведено аналитическое рассмотрение линейного взаимодействия лазерного пучка с нанообъектом сферической геометрии, построена модель на базе теории рассеяния Ми. Эта модель оказалась состоятельной для оценки распределения горячих пятен электрического поля на поверхности сферического нанокластера при взаимодействии с линейным и циркулярно-поляризованным излучением --- расхождения с численным расчётом оказались не более 3-5\% во время взаимодействия кластера с импульсом.

Также построена аналитическая модель генерации и затухания магнитного поля одиночным нанообъектом в результате взаимодействия с ультракоротким импульсом релятивистской интенсивности. Эти оценки были качественно и количественно проверены при помощи численного моделирования частиц в ячейках. Результаты моделирования показали, что модель генерации магнитного поля, описывающая его зависимость от начального радиуса, длительности и амплитуды циркулярно-поляризованного импульса, для тяжёлых ионов применима до интенсивности импульса порядка $10^{19} - 10^{20}$ W/cm$^2$ --- когда пондеромоторная сила импульса влияет на распределение ионов незначительно, а компрессия вдоль продольной оси мала. В этом случае мы действительно можем пренебречь движением ионов и рассматривать генерацию и эволюцию магнитного поля только в зависимости от движения электронов.

Для высоких интенсивностей было обнаружено значительное влияние движения ионов на процесс генерации магнитного поля. В частности, при пиковой интенсивности порядка $10^{21} - 10^{22}$ W/cm$^2$ пондеромоторная сила света становится достаточно сильна, чтобы импульс длительностью 10 fs сдвинул и деформировал нанокластер радиуса 200 nm на 25-30\%, а азимутальные силы, возникающие в результате взаимодействия тока в плазме и генерируемого магнитного поля, создают компрессию, что даёт в результате зону пространственного распределения ионов с пиковой ионной плотностью превышаюшей начальную в 10-15 раз. Это приводит к значительному увеличению пикового значения генерируемого магнитного поля, усиливает его локализацию. Были достигнуты значения 15-20 GGs магнитного поля в области локализации порядка 30-40 nm в азимутальной плоскости.

Полученные результаты и оценки планируется использовать в дальнейших исследованиях, связанных с генерацией экстремального магнитного поля при помощи множества кластеров и нанообъектов другой геометрии, находящихся в фокальной области лазерного пучка, а также взаимодействием между магнитными моментами кластеров и возбуждением вторичных колебаний и переизлучений.
        }

        \block[bodyoffsety=0.9cm]{Список литературы}{
            \begin{center}
                \mbox{}\vspace{-1\baselineskip}
                \printbibliography[heading=none] 
            \end{center}
        }

\end{columns}
\end{document}