В общем случае расчет взаимодействия мощного лазерного импульса с группой плотных сферических кластеров, находящихся в трехмерном пространстве, требует длительных и сложных нестационарных расчетов из-за изменения электронной плотности кластеров во времени. Чтобы проверить масштаб такого изменения, мы смоделировали эволюцию распределения электронной плотности для одиночного одномерного кластера с помощью LPIC++~\cite{Pfund1998}.

\begin{tikzfigure}
    \includegraphics[width=0.6\linewidth]{../img/lpic/htr_over_2a_a}\label{lpic_htr:image}\caption{Асимптотическое поведение средней общей толщины переходного слоя при $0 \leq t \leq 10T$ по отношению к радиусу мишени. $n_c$ используемое в построении, соответствует критической плотности для длины волны $\lambda = \lambda_{10}$.}
\end{tikzfigure}