Для проверки аналитической модели были рассчитано значение комплексного показателя преломления при $\lambda_{10} = 83$ нм, $ka = 0.7$. Мы рассматриваем случай, когда частота электрон-ионных столкновений $\nu_e$ много меньше частоты гармоники, поэтому взаимодействие можно считать бесстолкновительным~\cite{andreev_lecz}.

    \begin{tikzfigure}
        \subcaptionbox{$\lambda = \lambda_{L} = 830$ нм.}{
            \includegraphics[width=0.45\linewidth]{../img/mph_new/es_ka0.7_1harm}
        }
        \hfil
        \subcaptionbox{$\lambda = \lambda_{10} = 83$ нм.}{
            \includegraphics[width=0.45\linewidth]{../img/mph_new/es_ka0.7_10harm}
        }
        \label{ka0.7:image}\caption{$ka = 0.7$ ($a \approx 8.9$ нм); $|\vectbf{E}{s}|^2$ в плоскости поляризации падающей волны.}
    \end{tikzfigure}

В случае резонанса (Рис. 4а) амплитуда рассеянного поля вблизи кластера значительно выше, чем в отсутствие резонанса (Рис. 4б), когда рассеянные волны практически не наблюдаются.