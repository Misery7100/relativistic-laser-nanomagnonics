В пределах микрометровых длин волн фотонные кристаллы и решетки могут использоваться для направления или дифрагирования электромагнитных волн, в то время как аналогичные рентгеновские манипуляции могут использовать кристаллы с атомами, регулярно расположенными на расстоянии нескольких нанометров друг от друга, в качестве рассеивающих центров. В то же время большой разрыв между этими диапазонами длин волн, называемый XUV (экстремальным ультрафиолетом) или жестким ультрафиолетом, оказывается сложным для манипулирования. Для решения этой проблемы предлагается использовать массивы сферических нанокластеров для направленного рассеяния жесткого ультрафиолетового излучения (Рис.~\ref{intsch:image}).

\begin{tikzfigure}
    \includegraphics[width=0.75\linewidth]{../img/plasma_area2}\label{intsch:image}\caption{Схема взаимодействия. Плоскость поляризации параллельна одной из граней кубической области. Размеры сферических кластеров порядка нескольких нанометров, а расстояние между ними составляет не менее сотен нанометров. Распределение кластеров внутри кубической области в общем случае произвольно.}
\end{tikzfigure}