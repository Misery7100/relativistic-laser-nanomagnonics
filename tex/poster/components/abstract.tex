Периодические поверхностные решетки или фотонные кристаллы — отличные инструменты для управления светом. Однако этот метод менее эффективен в случае экстремального ультрафиолетового света из-за высокого поглощения любого материала в этом диапазоне частот. В работе исследуется возможность углового усиления такого излучения при помощи рассеяния на подходящих сферических кластерах. Была разработана аналитическая модель с использованием диэлектрической функции плазмы Друде и теории рассеяния Ми. Модель построена в квазистатическом приближении, так как время ионизации меньше длительности импульса, что значительно меньше времени разлета плазмы. Оценены резонансные параметры мишени по десятой гармонике титан-сапфирового лазера и найдено усиление рассеянного поля в резонансном случае по сравнению с первой гармоникой. Используя те же условия резонанса для одного кластера, мы моделируем дифракцию на массиве таких кластеров с помощью кода CELES. Полученные результаты показывают значительное усиление рассеянного поля в резонансном случае для больших углов, что соответствует теории дифракции Брэгга-Вульфа, --- возможность управления высокими гармониками лазерного излучения в XUV-диапазоне с помощью ионизированного кластерного газа.