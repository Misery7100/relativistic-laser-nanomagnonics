Условие дифракции в случае трехмерной регулярной решетки с упругим рассеянием~\cite{Kittel86} можно преобразовать следующим образом:

    \begin{equation}
        \begin{cases}
            \cos{\theta_0}\sin{\Delta \theta}\cos{\left( \Delta \varphi - \varphi_0 \right)} - \sin{\theta_0} \left( \cos{\Delta \theta} - 1 \right) = \cfrac{h^{\prime} \lambda}{d}
            \\
            \sin{\Delta \theta} \sin{\left( \Delta \varphi - \varphi_0 \right)} = \cfrac{k^{\prime} \lambda}{d}
            \\
            \sin{\theta_0}\sin{\Delta \theta}\cos{\left( \Delta \varphi - \varphi_0 \right)} + \cos{\theta_0} \left( \cos{\Delta \theta} - 1 \right)= \cfrac{l^{\prime} \lambda}{d}
        \end{cases}
        \label{bragg_wolf_order_spherical}
    \end{equation}
    \begin{equation*}
    \end{equation*}

\noindent где $\Delta \theta,\:\Delta \varphi$ --- углы, характеризующие отклонение направления дифрагированного излучения относительно падающего, $\theta_0,\:\varphi_0$ --- углы, характеризующие поворот мишени (решётки) в пространстве, $h^\prime,\:k^\prime,\:l^\prime$ --- индексы Миллера (Рис.~\ref{3ddiffr:image}), $\vectbf{e}{\textrm{in}} = \vectbf{e}{z}$ --- фиксированное направление падающего излучения, $d$ --- расстояние между центрами кластеров. Используя (\ref{bragg_wolf_order_spherical}), можно получить угловое распределение дифрагированного излучения при заданных начальных параметрах $d$, $\lambda$, $\theta_0$, $\varphi_0$.

\begin{tikzfigure}
    \subcaptionbox{Проекция на плоскость $xz$.}{
        \includegraphics[width=0.47\linewidth]{../img/article1_shortened_pic1}
    }
    \hfil
    \subcaptionbox{Проекция на плоскость $xy$.}{
        \includegraphics[width=0.47\linewidth]{../img/article1_shortened_pic2}
    }
    \label{3ddiffr:image}\caption{Общая схема взаимодействия падающего излучения с решеткой. $r_{\textrm{gas }}$ --- радиус газовой струи, представляющей мишень, $w$ --- ширина Гауссова пучка падающего излучения.}
\end{tikzfigure}