Рассмотрим одиночный кластер радиусом $a$, облучаемый коротким фемтосекундным импульсом с интенсивностью $I_{h} \approx 10^{14}$ $\textrm{Вт/см}^2$. Модель Друде дает представление о диэлектрической функции плазмы:

    \begin{equation}
		\varepsilon (\omega) = 1 - {\left( \frac{\omega_{pe}}{\omega} \right)}^2 \frac{1}{1+i \beta_{e}}, \qquad \omega_{pe} = \sqrt{\frac{4 \pi e^2 n_e}{m_e}},
		\label{eps_plasma}
    \end{equation}
    \begin{equation*} % artificial indent after the equation
    \end{equation*}

\noindent где $\omega$ --- частота рассматриваемой гармоники, $\omega_{pe}$ --- плазменная частота электронов, $n_e = Z n_i$ --- электронная плотность, $Z$ --- средняя степень ионизации, $n_i$ --- ионная плотность. $\beta_{e} = v_e / \omega$ и $v_e$ --- коэффициент электрон-ионных столкновений в приближении Спитцера.

\begin{tikzfigure}
  \includegraphics[width=0.7\linewidth]{../img/single_sph_scheme}\label{single_sph_scheme:image}\caption{Схема аналитической модели.}
\end{tikzfigure}

Теорию Ми можно использовать для описания рассеянного поля и поля внутри рассеивающего объекта. Рассмотрим сферический кластер и $x$-поляризованную плоскую волну, распространяющуюся вдоль оси $z$ (Рис.~\ref{single_sph_scheme:image}). Используя обобщенное разложение Фурье в случае изотропной среды, имеем следующий вид коэффициентов рассеянного поля~\cite{boren_huffman}:

    \begin{equation}
		\vectbf{E}{s} = \sum_{n = 1}^{\infty}E_n \left[ i a_n\left(ka, m\right) \vectbf{N}{}^{(3)}_{e1n} - b_n\left(ka, m\right) \vectbf{M}{}^{(3)}_{o1n} \right], \quad E_n = i^{n} E_0 \frac{2n + 1}{n \left(n + 1\right)},
        \label{E_s_sph}
    \end{equation}

    \begin{equation}
		a_n(x,\:m) = \frac{m \func{\psi}{n}{\prime}{x} \func{\psi}{n}{}{mx} - \func{\psi}{n}{\prime}{mx} \func{\psi}{n}{}{x}}{m \func{\xi}{n}{\prime}{x} \func{\psi}{n}{}{mx} - \func{\psi}{n}{\prime}{mx} \func{\xi}{n}{}{x}},
		\label{an_bessel}
    \end{equation}

    \begin{equation}
        b_n(x,\:m) = \frac{\func{\psi}{n}{\prime}{x} \func{\psi}{n}{}{mx} - m \func{\psi}{n}{\prime}{mx} \func{\psi}{n}{}{x}}{\func{\xi}{n}{\prime}{x} \func{\psi}{n}{}{mx} - m \func{\psi}{n}{\prime}{mx} \func{\xi}{n}{}{x}},
        \label{bn_bessel}
    \end{equation}
    \begin{equation*} % artificial indent after the equation
    \end{equation*}

\noindent где $\funccomp{\psi}{n}{}{\rho} = \rho \funccomp{j}{n}{}{\rho}$, $\funccomp{\xi}{n}{}{\rho} = \rho \funccomp{h}{n}{}{\rho}$ --- функции Риккати-Бесселя, $h_n = j_n + i \gamma_n$ --- сферические функции Ганкеля первого рода, $x = ka$ --- сферические функции Ганкеля первого рода, $ m = \sqrt{\varepsilon} $ --- комплексный показатель преломления.

% ============================= %

\begin{tikzfigure}
  \subcaptionbox{$ka = 0.5$, zero order approximation.}{
      \includegraphics[width=0.48\linewidth]{../img/sph_base/sph_ka0.5_123}
  }
  \hfil
  \subcaptionbox{$ka = 1.5$, first order approximation.}{
      \includegraphics[width=0.48\linewidth]{../img/sph_base/sph_ka1.5_123_1st}
  }
  \label{ab_asymp:image}\caption{Коэффициенты сферических гармоник в нулевом и первом приближении, $\beta_e = 0$. Кривые ``exact'' построены с использованием полных разложений.}
\end{tikzfigure}