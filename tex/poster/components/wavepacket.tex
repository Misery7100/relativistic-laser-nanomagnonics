\begin{tikzfigure}
    \subcaptionbox{Рассеяние 10-ой гармоники.}{
        \includegraphics[width=0.48\linewidth]{../img/celes/E_squared/eint_10harm_15deg_0.0nonreg.pdf}
    }
    \hfil
    \subcaptionbox{Рассеяние волнового пакета.}{
        \includegraphics[width=0.48\linewidth]{../img/celes/E_squared/eint_wavepacket2_15deg_0.0nonreg.pdf}
    }
    \label{wavepacket1:image}\caption{Диаграмма углового рассеяния гауссового волнового пакета и 10-ой гармоники. $\theta_0 = 15^\circ$, $\varphi_0 = 0^\circ$, $d = 2\lambda_{10}$, радиус кластеров $a = 20$ нм.}
\end{tikzfigure}

\begin{tikzfigure}
    \subcaptionbox{Рассеяние однорядным массивом кластеров.}{
        \includegraphics[width=0.48\linewidth]{../img/celes/plane_flat_to_compare.pdf}
    }
    \hfil
    \subcaptionbox{Рассеяние трёхрядным массивом кластеров.}{
        \includegraphics[width=0.48\linewidth]{../img/celes/3plane_flat_to_compare.pdf}
    }
    \label{cluster_rows:image}\caption{Рассеяние гармоники с $\lambda \approx 89$ нм массивом кластеров, $\varphi_0 = 0^\circ$, $\theta_0 = 30^\circ$, $a = 30$ нм, расстояние между центрами кластеров равно $d = 3\lambda$, падающее поле направлено от нижней от левого угла к правому верхнему под углом $\theta_0$ по отношению к нормали к мишени.}
\end{tikzfigure}