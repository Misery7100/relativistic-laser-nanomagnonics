\section{Нелинейное взаимодействие релятивистского лазерного пучка с нанообъектом и генерация магнитного поля}

\subsection{Аналитическое описание}

Взаимодействие циркулярно-поляризованного лазерного импульса с нанокластером приводит к закручиванию части электронов кластера, что приводит к возникновению продольного магнитного поля~(\autoref{interaction_magnetic:image}).

    \img[img/draftAAA/interaction.png]{Общая схема генерации магнитного поля~\cite{andr_plat_2021}.}{interaction_magnetic:image}{0.5\textwidth}

Рассмотрим пучок гармоник, падающий на одиночный кластер. Пусть кластер находится в каустике пучка, а его радиус мал по сравнению с шириной каустической зоны, тогда можем считать все приходящие на него гармоники плоскими волнами. Предположим, что гармоники в пучке не взаимодействуют друг с другом, так что итоговый результат будет представлять собой суперпозицию всех отдельных взаимодействий. Циркулярно-поляризованный импульс несёт момент, который поглощается электронами:

    \begin{equation}
        M_e = \eta M_L = \eta\frac{\EuScript{E}_L}{\omega_L},
    \end{equation}
    \begin{equation}
        \EuScript{E}_L = \pi r^2_{\textrm{eff}} \int_{0}^{\infty} I(t) dt = I_L \pi r^2_{\textrm{eff}}\tau_L,
        \label{Eps_L}
    \end{equation}

\noindent где $M_e$ --- часть момента, поглощённая электронами, $\omega_L$ --- частота гармоники в лазерном импульсе, с которой взаимодействует кластер, $\EuScript{E}_L$ --- часть энергии импульса, поглощённая электронами кластера, $I_L$ --- интенсивность лазерного импульса, $\tau_L$ --- длительность лазерного импульса, $r_{\textrm{eff}}$ --- эффективный радиус захвата, т.е. радиус, где концентрация электронов равна критической концентрации ($n_e(r_{\textrm{eff}}) = n_c = \omega_L^2 m_e / 4 \pi e^2$)~\cite{andr_plat_2021}. Для простоты мы рассматриваем прямоугольный временой профиль лазерного импульса. Из общих соображений инвариантности плотности электронов относительно объема можем оценить эффективный радиус захвата через начальный радиус кластера следующим образом:

    \begin{equation}
        r_{\textrm{eff}} = R \cdot \sqrt[3]{\frac{n_{e0}}{n_e}} < R \cdot \sqrt[3]{\frac{n_{e0}}{n_c}},
        \label{r_eff_R}
    \end{equation}

\noindent где $R$ --- начальный радиус кластера, $n_{e0}$ --- начальная электронная плотность кластера, $n_e$ --- электронная плотность кластера в результате взаимодействия с импульсом, $n_c$ --- критическая плотность, соответствующая частоте гармоники лазерного импульса. Магнитный момент, приобретаемый кластером, и генерируемое магнитное поле~\cite{andr_plat_2021}:

    \begin{equation}
        \mu_e = \frac{e}{m_e c \gamma} M_e = \frac{e}{m_e c \sqrt{1 + a^2}} M_e,
    \end{equation}
    \begin{equation}
        \vectbf{H}{}{} =  \cfrac{3\vectbf{r}{}{} \left( \vectbf{\mu}{e}{},\:\vectbf{r}{}{} \right)}{r^{5}} - \cfrac{\vectbf{\mu}{e}{}}{r^{3}} = \cfrac{2 \vectbf{\mu}{e}{}}{r_{\textrm{eff}}^{3}},
    \end{equation}
    \begin{equation}
        H = \frac{2 \mu_e}{r_{\textrm{eff}}^{3}} = \frac{2e \eta}{m_e c \gamma \omega_L} \frac{I_L \pi \tau_L}{r_{\textrm{eff}}}, \quad I_L = \frac{c E_L^2}{4\pi} \quad \Rightarrow
    \end{equation}
    \begin{equation}
        \frac{H}{E_L} = \eta \frac{a}{\sqrt{1 + a^2}} \cdot \frac{c \tau_L}{2 r_{\textrm{eff}}},
    \end{equation}

\noindent где $a = e E_L\:/\:m_e \omega_L c$ --- безразмерная амплитуда электрического поля лазерного импульса, $\gamma = \sqrt{1 + a^2}$ --- Лоренц-фактор электрона во внешнем электромагнитном поле~\cite{landau_field_theory}. Возьмём в рассмотрение предварительно ионизированный кластер из золота Au$^{+30}$, $Z = 30$. Тогда для начальной ионной плотности $n_{i0} = 6 \cdot 10^{22} \textrm{ cm}^{-3} \approx 35 n_c$ при $\lambda_L = 0.8$ $\upmu$m, $n_{e0} = Z n_{i0} = 35Z n_c$ при помощи (\ref{r_eff_R}) получаем:

    \begin{equation}
        \frac{H}{E_L} = \eta \frac{a}{\sqrt{1 + a^2}} \cdot \frac{c \tau_L}{20 R}.
        \label{H_EL_R}
    \end{equation}

Очевидно, что максимизация генерируемого магнитного поля возможна путём увеличения амплитуды лазерного импульса $E_L$, его длительности $\tau_L$, а также путём уменьшения начального радиуса кластера $R$. Увеличение амплитуды и длительности лазерного импульса имеет ограничения, связанное с разлётом кластера под действием кулоновских и тепловых сил:

    \begin{equation}
        E_L < E_{ion} = \frac{4}{3} \pi Z e n_{i0} R,
        \label{E_L_limit}
    \end{equation}
    \begin{equation}
        \tau_L \leq \tau_i = \frac{r_{\textrm{eff}}}{v_i},
        \label{tau_L_max}
    \end{equation}

\noindent то есть поле ионного остова кластера должно быть больше поля лазерного импульса, чтобы обеспечить отсутствие кулоновского взрыва, а длительность лазерного импульса не должна превышать время разлёта ионов $\tau_i$ за пределы эффективного радиуса захвата.

Для достаточно больших амплитуд поля $a$ можем построить верхнюю оценку генерируемого магнитного поля, используя~(\ref{tau_L_max}):

    \begin{equation}
        \frac{H_{\max}}{E_L} = \frac{\eta c}{2 v_i},
    \end{equation}

    \begin{equation}
        v_i = \sqrt{\frac{Z T_e}{m_i}} = \sqrt{\frac{Z m_e c^2 \left( \sqrt{1 + a^2} - 1 \right)}{m_i}} \approx c \sqrt{Z a \frac{m_e}{m_i}} \Rightarrow
    \end{equation}

    \begin{equation}
        \frac{H_{\max}}{E_L} = \frac{\eta}{2 \sqrt{a}} \sqrt{\frac{m_i}{Z m_e}},
        \label{H_max_plat}
    \end{equation}

\noindent где $m_i$ --- масса иона нанокластера, $v_i$ --- скорость ионного звука. Такая оценка в применении к кластеру из Au$^{+30}$ даёт $\ifrac{H_{\max}}{E_L} = 3 \eta$, что говорит о возможности генерации магнитных полей, превышающих по амплитуде падающий лазерный импульс при коэффициенте поглощения $\eta > \ifrac{1}{3}$. Необходимо отметить, что в этих оценках магнитное поле принимается однородной величиной во всей области, ограниченной эффективным радиусом захвата, но на самом деле магнитное поле может быть локализовано в более узкой области. Также очевидно, что взаимодействие неоднородно в области, ограниченной эффективным радиусом захвата, а поглощение происходит в сферическом слое, внутреннюю границу которого мы можем оценить как $r_{\textrm{abs}0} = r_{\textrm{eff}} - R$. Электроны этого слоя поглощают излучение и генерируют магнитное поле. Пусть эффективная область локализации магнитного поля ограничена некоторым $r_{\textrm{loc}} < r_{\textrm{eff}}$, тогда:

    \begin{equation}
        \cfrac{\displaystyle\int_{0}^{r_{\textrm{loc}}} r^2 H(r) dr}{\displaystyle\int_{r_{\textrm{abs}0}}^{r_{\textrm{eff}}} r^2 E_L(r) dr}  = \frac{\eta }{2 \sqrt{a}} \sqrt{\frac{m_i}{Z m_e}} \quad \Rightarrow 
    \end{equation}
    \begin{equation}
        \cfrac{H_{\max}}{E_L} = \frac{\eta }{2 \sqrt{a}} \frac{r_{\textrm{eff}}^3 - r_{\textrm{abs}0}^3}{r_{\textrm{loc}} ^ 3} \sqrt{\frac{m_i}{Z m_e}},
        \label{h_est_prec} 
    \end{equation}

\noindent что при $r_{\textrm{loc}} = \ifrac{r_{\textrm{eff}}}{2}$ даёт уже $\ifrac{H_{\max}}{E_L} \approx 13 \eta$.

Затухание магнитного поля можно оценить при помощи следующего выражения:

    \begin{equation}
        H(t) = H(0) \cdot \cfrac{R^3(0)}{R^3(t)}\cdot e^{-\gamma_d t},
        \label{H_evo_inva}
    \end{equation}

\noindent полагая, что при расширении кластера его полный механический и магнитный момент инвариантен с учётом поправки на релаксацию поля в результате переизлучения магнитным диполем в пространство, описываемого коэффициентом затухания $\gamma_d$, при этом намагниченность падает с ростом объема кластера. Вместо зависимости радиуса кластера от времени $R$ здесь можно использовать радиус области эффективной локализации магнитного поля из (\ref{h_est_prec}), а её расширение описывать комбинацией моделей теплового и кулоновского расширения~\cite{andr_plat_2021}.

\subsection{PIC моделирование}

Для численного моделирования был использован код EPOCH, широко использующийся в области физики плазмы для высокопроизводительного крупномасштабного моделирования лазер-плазменных взаимодействий~\cite{epochpic}.

Метод частиц в ячейках (PIC), используемый EPOCH, позволяет самосогласованно обрабатывать заряженные частицы и электромагнитные поля. Код делит область моделирования на сетку, где каждая ячейка рассматривается как частица, которая испытывает силы и движение из-за электромагнитных полей. Эти поля обновляются на каждом временном шаге в зависимости от движения частиц, гарантируя, что динамика частиц и электромагнитные поля взаимно влияют друг на друга самосогласованным образом.

В рамках рассматриваемой задачи был рассмотрен одиночный Au$^{+30}$ кластер с радиусом в диапазоне $R = 50\:\dots\:200$ nm, взаимодействующий с лазерным импульсом с гауссовым временным профилем, длительностью $\tau_L = 10$ fs, длиной волны $\lambda_L = 0.8$ $\upmu$m и интенсивностью в диапазоне $I_L = 10^{19}\:\dots\:10^{22}$ W/cm$^{2}$. Мишень в предварительно ионизированном состоянии была задана макрочастицами: 200 электронами и 40 ионами в ячейке соответственно, электронная плотность $n_e = 1600 n_c$, ионная плотность $n_i = \ifrac{n_e}{Z}$ --- таким образом общее число макрочастиц, используемое для представления плазмы, порядка $11 \cdot 10^6$. Взят бокс моделирования $4\times4\times4$~$\upmu$m$^3$ с разрешением $400\times400\times400$ ячеек, применены CPML граничные условия~\cite{cpml2000} ввиду того, что радиус траектории электронов достаточно быстро выходит за пределы характерного размера бокса.

Зависимость продольной компоненты и модуля магнитного поля от времени на~\autoref{R200_H_Hx:image} хорошо демонстрирует общую динамику лазер-плазменного взаимодействия. Можно заметить, что до момента времени $t = 27.5$ fs модуль магнитного поля превышает модуль продольной компоненты приблизительно на 3-4 GGs, что отвечает времени полного прохождения лазерного импульса через бокс моделирования. Далее вплоть до $t = 40$ fs магнитное поле представлено только продольной компонентой $H_x$ и достигает второго максимума, что связано со стабилизацией магнитного диполя, образованного в результате лазер-плазменного взаимодействия~(\autoref{R200_Hxmax2:image}). После этого образованный диполь в результате остаточного вращения электронов и постепенного теплового и кулоновского разлёта трансформируется в квадруполь, поле которого уже не является строго продольным~(\autoref{R200_Hxmax3:image}).  

    \img[img/pic_new/R200_H_Hx_2.5fs]{Эволюция максимумов модуля продольной компоненты и полного модуля магнитного поля для кластера радиусом $R = 200$ nm при взаимодействии с импульсом $I_L = 10^{22}$ W/cm$^{2}$, $\tau_L = 10$ fs.}{R200_H_Hx:image}{0.55\textwidth}

Область эффективной локализации генерируемого магнитного поля имеет радиус порядка 200-250 nm, а пиковое значение, достигаемое после ухода лазерного импульса из бокса моделирования примерно в 2 раза превышает электрическое поле импульса, что согласуется с аналитикой, описанной ранее~(\ref{h_est_prec}).

    \img[img/pic_new/R200_H_decay_2.5fs_exp]{Эволюция продольной компоненты и модуля магнитного поля для кластера радиусом $R = 200$ nm при взаимодействии с импульсом $I_L = 10^{22}$ W/cm$^{2}$, $\tau_L = 10$ fs.}{R200_H_Hx:image}{0.55\textwidth}

Также области затухания магнитного поля в результате ухода лазерного импульса из бокса моделирования и в результате окончательной деградации наведенного магнитного мультиполя с хорошей точностью описываются моделью~(\ref{H_evo_inva}).

Распределения компонент магнитного поля на~\autoref{R200_Hxmax2:image} и \autoref{R200_Hxmax3:image} представлены в моменты времени, соответствующие максимумам полного модуля магнитного поля после ухода лазерного импульса из бокса моделирования.

Первый максимум, как было сказано ранее, соответствует наведению магнитного диполя, что отлично прослеживается на~\autoref{R200_Hxmax2:a} по наиболее интенсивным пятнам магнитного поля компоненты $H_x$. При этом также можно заметить протяжённую область вокруг магнитного диполя, имеющую небольшое положительное значение амплитуды магнитного поля, образование которой связано с остаточным вращением электронной оболочки и наведением вторичных мультипольных моментов, что также видно на~\autoref{R200_Hxmax2:c}. Второй максимум отвечает магнитному квадруполю, структура которого наилучшим образом прослеживается на~\autoref{R200_Hxmax3:b}. Модуль магнитного поля в этом случае даже выше за счёт большей локализации поля.

\begin{figure}[H]
    \subimgtwo[img/pic_new/R200_Hxmax2.png]{Продольная компонента магнитного поля $H_x$.}{R200_Hxmax2:a}{0.72\textwidth}
    \hfil
    \subimgtwo[img/pic_new/R200_Hymax2.png]{Поперечная компонента магнитного поля $H_y$.}{R200_Hxmax2:b}{0.72\textwidth}
    \subimgtwo[img/pic_new/R200_Hzmax2.png]{Поперечная компонента магнитного поля $H_z$.}{R200_Hxmax2:c}{0.72\textwidth}
    \caption{Компоненты магнитного поля в центральных сечениях пространства в момент времени $t = 37.5$ fs для кластера радиусом $R = 200$ nm при взаимодействии с импульсом $I_L = 10^{22}$ W/cm$^{2}$, $\tau_L = 10$ fs.}\label{R200_Hxmax2:image}
\end{figure}

\begin{figure}[H]
    \subimgtwo[img/pic_new/R200_Hxmax3.png]{Продольная компонента магнитного поля $H_x$.}{R200_Hxmax3:a}{0.72\textwidth}
    \hfil
    \subimgtwo[img/pic_new/R200_Hymax3.png]{Поперечная компонента магнитного поля $H_y$.}{R200_Hxmax3:b}{0.72\textwidth}
    \subimgtwo[img/pic_new/R200_Hzmax3.png]{Поперечная компонента магнитного поля $H_z$.}{R200_Hxmax3:c}{0.72\textwidth}
    \caption{Компоненты магнитного поля в центральных сечениях пространства в момент времени $t = 45$ fs для кластера радиусом $R = 200$ nm при взаимодействии с импульсом $I_L = 10^{22}$ W/cm$^{2}$, $\tau_L = 10$ fs.}\label{R200_Hxmax3:image}
\end{figure}

\subsection{Связь с моделью рассеяния Ми}

Полученные в результате PIC моделирования траектории электронных сгустков хорошо соответствуют результатам, полученным при помощи линейной модели, основанной на теории Ми.

\img[img/pic_old/linear_vs_pic_scat.png]{Азимутальный угол вылета электронного сгустка для кластера радиусом $R = 200$ nm при взаимодействии с импульсом $I_L = 10^{21}$ W/cm$^{2}$, $\tau_L = 10$ fs в сранении с результатами, полученными при помощи линейной модели.}{linear_vs_pic_scat:image}{0.55\textwidth}

В полученной при помощи численного моделирования зависимости можно проследить некоторую гармоническую составляющую зависимости азимутального угла $\theta$ от времени, которая связана с продольными колебаниями электронов в электромагнитном поле ультрарелятивистской интенсивности~\cite{landau_field_theory}.

\subsection{Оптимальные параметры генерации магнитного поля}

Как было описано ранее при помощи~(\ref{H_EL_R}, \ref{H_max_plat}), существуют параметры, влияющие на максимальную амплитуду генерируемого в результате взаимодействия магнитного поля. Результаты численного моделирования подтверждают описанную в~(\ref{H_EL_R}) зависимость от амплитуды падающего импульса вкупе с ограничением~(\ref{E_L_limit}) на максимально возможную амплитуду электрического поля импульса при заданном радиусе.

На~\autoref{Hx_by_intens:image} построены результаты оптимизации по радиусу и интенсивности лазерного импульса. Области резкого спада продольного магнитного поля соответствуют интенсивностям, которые при соответствующем радиусе кластера вызывают кулоновский взрыв, унося за собой все электроны. Видно, что для малого радиуса $R = 50$ nm генерируемое магнитное поле не превышает магнитное поле, которое несёт лазерный импульс (порядка 4.5 GGs).

Оптимальным радиусом при общем диапазоне интенсивностей оказался $R = 200$ nm, обеспечивающий почти четырёхкратное усиление магнитного поля для предельного значения интенсивности $I_L = 10^{22}$ W/cm$^{2}$.

    \img[img/pic_old/Hx_by_intens.png]{Пиковая амплитуда продольного магнитного поля в зависимости от интенсивности импульса при длительности $\tau_L = 10$ fs и радиуса кластера.}{Hx_by_intens:image}{0.65\textwidth}

Кроме интенсивности импульса, была также рассмотрена зависимость магнитного поля от длительности импульса $\tau_L$. Для относительно малых интенсивностей $I_L$ ($10^{19} - 10^{20}$ W/cm$^{2}$ для радиуса $R = 200$ nm) падающего импульса отклик кластера можно считать зависящим от энергии, переданной от импульса кластеру, $\EuScript{E}_L$ в целом, что также описывается в~(\ref{Eps_L}). В этом случае не происходит значительной деформации изначальной формы распределения ионной и электронной плотности под воздействием пондеромоторной силы. Можно сказать, что в этом случае генерируемое магнитное поле в некоторых пределах практически линейно зависит от длительности лазерного импульса $\tau_L$. 

При этом для больших значений интенсивности $I_L$ ($10^{21} - 10^{22}$ W/cm$^{2}$ для радиуса $R = 200$ nm) пондеромоторные силы значительным образом деформируют форму распределения ионной и электронной плотности, что при увеличении длительности лазерного импульса постепенно приводит к деградации структуры токов, рождающихся в плазме в результате взаимодействия, удалению слишком большой доли электронов за короткий промежуток времени, что значительно снижает выход магнитного поля и препятствует образованию магнитных мультиполей.

\subsection{Влияние движения ионов}

В результате численных экспериментов были получены данные, показывающие сложное поведение ионов под воздействием ультрарелятивистских импульсов фемтосекундной длительности. Ионы кластера в процессе взаимодействия разделяются на две зоны --- зону разлёта и зону компрессии~(\autoref{ni_sdf_r200:image})~\cite{ncfm2023}.

    \img[img/draftAAA/coloumb_thermal.png]{Разлёт Au$^{+30}$ кластера с $R = 100$ nm при $\tau_L = 10$ fs, $I_L = 10^{22}$ W/cm$^{2}$. Чёрная линия соответствует модели теплового расширения, зелёная --- кулоновскому расширению, синие точки --- результаты численного моделирования.}{coloumb_thermal:image}{0.55\textwidth}

\begin{figure}[H]
    \subimgtwo[img/pic_old/ni_sdf15_r200_2.png]{$t = 25$ fs.}{ni_sdf_r200:a}{0.49\textwidth}
    \hfil
    \subimgtwo[img/pic_old/ni_sdf16_r200_2.png]{$t = 27.5$ fs.}{ni_sdf_r200:b}{0.49\textwidth}
    \caption{Ионная плотность в объеме бокса моделирования, хорошо видно разделение ионов на две зоны: зону разлёта и зону компрессии. $R = 200$ nm, $I_L = 10^{22}$ W/cm$^{2}$, $\tau_L = 10$ fs.}\label{ni_sdf_r200:image}
\end{figure}

Поведение ионов, принадлежащих к зоне разлёта, в процессе всего взаимодействия с импульсом соответствует аналитической модели, описанной ранее~(\ref{H_EL_R}, \ref{H_evo_inva}). Движение ионов этой зоны соответствует комбинации моделей кулоновского и теплового разлёта, что можно пронаблюдать, построив зависимость внешнего края зоны от времени~(\autoref{coloumb_thermal:image}). Ионы, принадлежащие зоне компрессии, не отвечают описанию~(\ref{H_evo_inva}) во время взаимодействия с падающим импульсом и некоторое время после его ухода, когда стабилизируется наведенный магнитный диполь и магнитное поле достигает своего второго максимума.

Электроны связаны с ионами сильными электростатическими силами, поэтому их движение в области компрессии сильно скоррелировано с движением ионов. При этом часть закрученных лазерным импульсом электронов экстрагируется, образуя электронный сгусток, который двигается по спиральной траектории, как было описано ранее~\cite{laura2015}. Этот электронный сгусток создаёт спиральную ударную волну ионной плотности, которая лучше всего наблюдается в начале лазер-плазменного взаимодействия~(\autoref{ni_yz_t10.0:image}). Ударная волна способствует формированию двух зон ионов, так как её "голова" направлена внутрь, в то время как хвост раскидывает ионы от центра кластера, причём сам процесс квазиравномерен в азимутальной плоскости за счёт вращения электронного сгустка и формируемой ударной волны вслед за ним. Разделение так же хорошо прослеживается на фазовых диаграммах ионов~(\autoref{phase_R200:image}).

    \img[img/pic_old/ni_yz_t10.0.png]{Ионная плотность в поперечном $yz$ сечении Au$^{+30}$ кластера с $R = 100$ nm при $\tau_L = 10$ fs, $I_L = 10^{22}$ W/cm$^{2}$ в момент времени $t = 10$ fs.}{ni_yz_t10.0:image}{0.4\textwidth}

    \begin{figure}[H]
        \subimgtwo[img/new/x_px_R200.png]{Фазовая диаграмма $(x, v_x)$.}{phase_R200:a}{0.4\textwidth}
        \hfil
        \subimgtwo[img/new/y_py_R200.png]{Фазовая диаграмма $(y, v_y)$.}{phase_R200:b}{0.4\textwidth}
        \\
        \subimgtwo[img/new/z_pz_R200.png]{Фазовая диаграмма $(z, v_z)$.}{phase_R200:b}{0.4\textwidth}
        \caption{Фазовые диаграммы ионов по декартовым осям в момент максимума модуля магнитного поля, $R = 200$ nm, $\tau_L = 10$ fs, $I_L = 10^{22}$ W/cm$^{2}$.}\label{phase_R200:image}
    \end{figure}

Таким образом, эволюцию поведения ионов в результате взаимодействия с импульсом можно условно разделить на две части. Во время первой части ионы разделяются на две зоны в результате совокупного действия пондеромоторной силы со стороны падающего импульса и азимутально симметричной силы, связанной с генерируемым магнитным полем и наведенным током в плазме, поведение ионов разных зон описывается разными моделями, процесс происходит до тех пор, пока продольное магнитное поле не достигает своих следующего максимума. После этого наступает вторая часть, которая соответствует распаду магнитного поля в соответствии с моделью теплового и кулоновского расширения --- неоднородное распределение ионов и электронов расширяется практически изотропно, а магнитное поле падает в соответствии с моделью~(\ref{H_evo_inva}).

% И пондеромоторная сила, и азимутальная сила --- обе связаны с массой ионов, из которых состоит кластер, а значит характер сложного движения ионов будет также сильно скоррелирован с массой ионов, образуя два граничных случая относительно интенсивности падающего импульса: для сверхтяжёлых ионов поведение и движение будет полностью описываться радиальным разлётом нагретых в результате взаимодействия с внешним импульсом ионов и вращением электронной оболочки; для сверхлёгких ионов пондеромоторная сила будет достаточно велика, чтобы препятствовать какому-либо образованию структурированных токов и генерации магнитного поля в области кластера и целиком разрушить кластер сродни кулоновскому взрыву.

% Сравниваем с моделированием, когда ионы заморожены, сравнение для разных масс ионов (??)