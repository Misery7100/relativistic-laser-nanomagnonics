\section{Заключение}

Было проведено аналитическое рассмотрение линейного взаимодействия лазерного пучка с нанообъектом сферической геометрии, построена модель на базе теории рассеяния Ми. Эта модель оказалась состоятельной для оценки распределения горячих пятен электрического поля на поверхности сферического нанокластера при взаимодействии с линейным и циркулярно-поляризованным излучением --- расхождения с численным расчётом оказались не более 3-5\% во время взаимодействия кластера с импульсом.

Также построена аналитическая модель генерации и затухания магнитного поля одиночным нанообъектом в результате взаимодействия с ультракоротким импульсом релятивистской интенсивности. Эти оценки были качественно и количественно проверены при помощи численного моделирования частиц в ячейках. Результаты моделирования показали, что модель генерации магнитного поля, описывающая его зависимость от начального радиуса, длительности и амплитуды циркулярно-поляризованного импульса, для тяжёлых ионов применима до интенсивности импульса порядка $10^{19} - 10^{20}$ W/cm$^2$ --- когда пондеромоторная сила импульса влияет на распределение ионов незначительно, а компрессия вдоль продольной оси мала. В этом случае мы действительно можем пренебречь движением ионов и рассматривать генерацию и эволюцию магнитного поля только в зависимости от движения электронов.

Для высоких интенсивностей было обнаружено значительное влияние движения ионов на процесс генерации магнитного поля. В частности, при пиковой интенсивности порядка $10^{21} - 10^{22}$ W/cm$^2$ пондеромоторная сила света становится достаточно сильна, чтобы импульс длительностью 10 fs сдвинул и деформировал нанокластер радиуса 200 nm на 25-30\%, а азимутальные силы, возникающие в результате взаимодействия тока в плазме и генерируемого магнитного поля, создают компрессию, что даёт в результате зону пространственного распределения ионов с пиковой ионной плотностью превышаюшей начальную в 10-15 раз. Это приводит к значительному увеличению пикового значения генерируемого магнитного поля, усиливает его локализацию. Были достигнуты значения 15-20 GGs магнитного поля в области локализации порядка 30-40 nm в азимутальной плоскости.

Полученные результаты и оценки планируется использовать в дальнейших исследованиях, связанных с генерацией экстремального магнитного поля при помощи множества кластеров и нанообъектов другой геометрии, находящихся в фокальной области лазерного пучка, а также взаимодействием между магнитными моментами кластеров и возбуждением вторичных колебаний и переизлучений.