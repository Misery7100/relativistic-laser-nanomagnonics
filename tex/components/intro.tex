\section{Введение}

Лазерно-плазменные взаимодействия представляют собой яркую и быстро развивающуюся область исследований. По своей сути эта область исследует множество явлений, возникающих при взаимодействии интенсивного лазерного импульса с плазмой --- квазинейтрального газа заряженных и нейтральных частиц, проявляющего коллективное поведение~\cite{chen1987introduction}.

Фундаментальные концепции, лежащие в основе лазерно-плазменных взаимодействий, включают электромагнитные и электростатические волны, нестабильность параметров, лазерный синтез, ускорение заряженных частиц и гамма-лучи. Физика этих взаимодействий имеет решающее значение для различных приложений, связанных с интенсивными лазерными импульсами; сфера охватывает ключевые аспекты физики лазерной плазмы, включает в себя изучение взаимодействия лазерного излучения с веществом в экстремальных условиях, исследования термоядерного синтеза с инерционным удержанием, частиц лазерной плазмы и источников излучения, таких как ускорители лазерной плазмы, когерентные источники света в рентгеновском и крайнем ультрафиолетовом диапазонах, генерация фемтосекундных и аттосекундных импульсов излучения и многое другое~\cite{gibbon1996lpi}.

Одним из наиболее поразительных результатов исследования лазерно-плазменных взаимодействий является генерация экстремальных магнитных полей~\cite{andreev2021generationmd}. При правильно подобранных параметрах излучения и мишени коллективное движение и поведение электронов и ионов в плазме может приводить к созданию интенсивных магнитных полей, представляющих значительный научный интерес, поскольку они открывают новые возможности для технического прогресса, в частности для развития контролируемого лазерного термоядерного синтеза~\cite{miyamoto_controlfusion}. Релятивистские эффекты, подразумеваемые термином «ультрарелятивистский» в контексте лазерных импульсов, зачастую играют решающую роль в этих взаимодействиях. При использовании лазерных импульсов необычайной интенсивности электроны в плазме могут быть ускорены до скоростей, близких к световым. Возникающее в результате релятивистское движение этих электронов является ключом к генерации и динамике магнитных полей, возникающих в результате этих взаимодействий. %~\cite{andr_plat_2021}.

Как правило, в исследованиях лазер-плазменных взаимодействий используются обычные твёрдые или газовые мишени, которые имеют свои преимущества и особенности. В частности, твёрдые мишени, которые чаще всего представляют собой тонкую фольгу или плёнку, могут выдерживать лазерные импульсы значительной интенсивности без сильных повреждений, они часто используются в экспериментах, где необходимы высокие плотности и выделение большой энергии, например, в лазерном ускорении ионов~\cite{chen1987introduction}. С другой стороны, газовые мишени обеспечивают высокую степень контроля над начальными условиями лазерно-плазменного взаимодействия. Плотность и состав газа можно хорошо контролировать, а мишень пополнять между выстрелами. Газовые мишени используются при генерации высокочастотного излучения, ускорении электронов, а также в исследованиях генерации высших гармоник.

И хотя твёрдые и газовые мишени играют важную роль в изучении множества явлений лазерно-плазменных взаимодействий, существуют определенные задачи, в рамках которых такие мишени могут оказаться не лучшим выбором. В случае твердых целей материал высокой плотности может привести к увеличению отражательной способности и снижению пропускания, особенно для высокочастотного излучения, такого как экстремальное ультрафиолетовое (XUV) или рентгеновское излучение~\cite{boren_huffman}. Это может ограничивать эффективность ввода энергии лазерного импульса в мишень и снижать выход высокочастотного излучения. Кроме того, высокая плотность мишени также может привести к быстрой термализации и распространению энергии, что может предотвратить формирование локализованных высокоэнергетических явлений. Для газовых мишеней одна из основных проблем заключается в их низкой плотности. Это может привести к недостаточному поглощению лазерного импульса, особенно для импульсов высокой интенсивности или высокой частоты. Более того, в тех случаях, когда требуется распространение коротковолнового излучения, такого как XUV свет, излучение может значительно поглощаться газовой средой из-за фотоионизации, препятствуя эффективному прохождению и дифракции~\cite{boren_huffman}.

В свете этих ограничений многообещающей альтернативой становятся кластерные мишени~\cite{Krainov2000}. Такие мишени представляют собой уникальный баланс между характеристиками высокой плотности твердых мишеней и характеристиками низкой плотности газовых мишеней. Эта комбинация свойств позволяет эффективно поглощать энергию и генерировать высокий уровень заряда, что может быть особенно полезным для целого ряда приложений, включая генерацию высоких гармоник и магнитных полей высокой амплитуды.

Всё это вкупе приводит нас к такому явлению как релятивистская лазерно-плазменная нано-магноника. Она представляет собой развивающуюся область на пересечении физики лазерной плазмы, нанотехнологий и магнетизма. Его название намекает на ключевые элементы этой дисциплины. «Релятивистский» относится к скоростям, с которыми работают частицы и поля, участвующие в этом процессе, — скоростям, близким к скорости света, «лазер» указывает на использование высокоинтенсивных лазерных импульсов в качестве основного инструмента для изучения и управления рассматриваемыми системами, «нано» относится к наномасштабу, в частности, к использованию наноразмерных кластеров в качестве мишеней для лазерных импульсов. Наконец, «магноника» относится к изучению магнонов, элементарных возбуждений магнитных систем и их взаимодействий с другими компонентами этой системы.

Магноны, как упоминалось ранее, представляют собой квазичастицы, представляющие состояния коллективного возбуждения в магнитной системе. Эти возбуждения соответствуют спиновым волнам или изменениям магнитной ориентации атомов внутри материала. В наномасштабе эти возбуждения могут демонстрировать уникальное поведение и взаимодействие, которых нет в более крупных объемных материалах. Таким образом, магнитные поля, генерируемые в результате взаимодействия магнонов системы между собой, могут иметь уникальные свойства и экстремально высокую амплитуду~\cite{andreev2021generationmd}. 

Хотя магноника изучает коллективное взаимодействие магнонов, в данной работе мы рассматриваем генерацию экстремального магнитного поля одиночным кластером, поэтому строго назвать это магноникой нельзя. Но определенно можно сказать, что результаты, оценки и зависимости, полученные в этой работе, будут использованы при исследовании и расчётах систем из множества кластеров.

Далее мы углубимся в конкретные модели и методологии, используемые для исследования описанных явлений. Мы рассмотрим модель рассеяния Ми ---  классическую теорию, применимую для оценки горячих точек электрического поля, которые соответствуют генерации электронных сгустков при взаимодействии нанокластера с лазерным импульсом. Также мы опишем модель генерации магнитного поля одиночным кластером, основанную на движении электронов в плазме, сопоставим результаты трёхмерного моделирования частиц в ячейках (PIC) с этими моделями, проанализируем параметры генерации магнитных полей, а также характер движения электронов и ионов под воздействием ультрарелятивистки интенсивных фемтосекундных лазерных импульсов.