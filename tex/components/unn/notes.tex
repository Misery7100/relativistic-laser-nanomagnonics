\section{Notes}

The interaction of a relativistic laser pulse with a charged cluster can lead to complex dynamics and phenomena. The spiral motion of ions you observed is likely due to a combination of effects, such as the ponderomotive force from the laser field, the electrostatic interactions between ions, and the magnetic field induced by the movement of electrons.

The spiral motion of ions can impact the magnetic field yield and the motion of electrons in several ways:

\begin{itemize}
    \item The motion of ions may alter the effective charge distribution within the cluster, leading to a redistribution of the magnetic field.
    \item The spiral motion can induce time-varying magnetic fields, which can, in turn, affect the motion of electrons through the Lorentz force.
    \item The varying distribution of ions and associated magnetic fields can also modify the overall energy coupling between the laser pulse and the cluster, affecting the overall field dynamics.
\end{itemize}

As for comparing geometries, a cylindrical structure might indeed yield a stronger longitudinal magnetic field. The key advantage of a cylindrical geometry is that it can better confine the magnetic field lines within the structure, leading to a more significant enhancement of the longitudinal component. The optimal parameters for the cylindrical structure would depend on the specific goals of your research and the desired magnetic field properties. To compare the performance of the 100 nm Au cluster with a cylindrical geometry, you can run numerical simulations for different cylinder dimensions and analyze the resulting magnetic field profiles.

In summary, the spiral motion of ions can impact the magnetic field yield and electron dynamics in several ways, and a cylindrical geometry may provide better confinement of the longitudinal magnetic field. To optimize the geometry for your specific goals, I recommend running numerical simulations with various parameters and comparing the results.

---

If the spiral motion of ions is caused by density shockwaves from the ponderomotive light field, the impact on magnetic field yield and electron dynamics can be more complex:

\begin{itemize}
    \item The density shockwave can cause a periodic compression and rarefaction of the plasma, which can result in a local enhancement of the magnetic field during the compression phase.
    \item The density variation can also cause changes in the local plasma frequency, affecting the coupling between the laser pulse and the plasma, and potentially altering the energy transfer and magnetic field generation.
    \item The motion of ions caused by the density shockwave can still induce time-varying magnetic fields, which can affect electron motion through the Lorentz force.
\end{itemize}

To compare a cylinder with a cluster, you can start by matching the effective parameter (100 nm) by setting the cylinder's radius to 100 nm. You can then investigate the behavior for various cylinder lengths, keeping in mind that different lengths will result in different aspect ratios and confinement properties.

The better confinement of magnetic field lines in a cylindrical geometry arises due to its shape and the distribution of charges within the structure. In a cylinder, the field lines are primarily constrained to run along the length of the cylinder, promoting a stronger longitudinal magnetic field. In contrast, the field lines in a spherical cluster have more freedom to spread out in all directions, which can lead to a weaker longitudinal field.

As for the comparison, you could set the initial goal to achieve the same magnetic field yield as a base point. By comparing the magnetic field profiles in different geometries, you can identify the configurations that provide the desired field properties. Once you have found the optimal cylindrical geometry, you can then investigate other properties like energy coupling, field confinement, and electron dynamics to make a more comprehensive comparison between the two geometries.

---

I'll provide a brief overview of magnetic field lines in a cylinder and a cluster, as well as a summary of magnetic field generation via the interaction of a relativistic intensity laser pulse with a gold cluster. If you'd like to share the article reference, I'd be happy to refine my explanations further based on that information.

Magnetic field lines in a cylinder and a cluster:
In a cylinder, the magnetic field lines predominantly run parallel to the cylinder's axis, confined within the structure. The cylindrical geometry encourages the formation of a more substantial longitudinal magnetic field component. This confinement is due to the spatial distribution of the charges and the symmetry of the cylindrical geometry.

On the other hand, in a spherical or quasi-spherical cluster, the magnetic field lines can spread out in all directions. This distribution leads to a weaker longitudinal magnetic field component, as the field lines tend to be more isotropic.

Magnetic field generation via radiation of a gold cluster with a relativistic intensity laser pulse:
When a relativistic intensity laser pulse interacts with a gold cluster, several processes can contribute to the generation of a magnetic field. Some of the primary mechanisms include:


a. Inverse Faraday effect: Circularly polarized laser pulses can induce a magnetic field through the inverse Faraday effect. The circular polarization imparts an angular momentum to the electrons, resulting in a net magnetic moment. The magnetic field generated by this effect can be described by:

\begin{equation}
    B = - \frac{c^2 \mu_0}{2 \omega} \cdot \left[ E \times H \right]
\end{equation}

where $B$ is the magnetic field, $c$ is the speed of light, $\mu_0$ is the vacuum permeability, $\omega$ is the angular frequency of the laser light, $E$ is the electric field, and $H$ is the magnetic field of the laser pulse.

b. Ponderomotive forces and plasma currents: The laser-induced ponderomotive forces can drive electrons to form plasma currents, which in turn generate magnetic fields through Ampere's circuital law:

\begin{equation}
    \nabla \times B = \mu_0 J
\end{equation}

where $B$ is the magnetic field, $\mu_0$ is the vacuum permeability, and $J$ is the current density.

c. Biermann battery effect: The laser-cluster interaction can cause gradients in the plasma's electron density and temperature. These gradients can result in the generation of magnetic fields through the Biermann battery effect:

\begin{equation}
    B \sim \nabla n_e \times \nabla T_e
\end{equation}

where $B$ is the magnetic field, $n_e$ is the electron density, and $T_e$ is the electron temperature.

These processes are not exhaustive, and other mechanisms, such as plasma instabilities and magnetic reconnection, can also contribute to magnetic field generation during the interaction between a relativistic intensity laser pulse and a gold cluster.

If you'd like to provide the article reference or further information, I'd be happy to refine and expand on these explanations based on that content.

---

Ponderomotive forces are non-linear forces experienced by charged particles in the presence of an oscillating electromagnetic field, such as those created by a laser pulse. The ponderomotive force acts on the charged particles, typically electrons, to drive them out of regions with high-intensity fields. It is a slowly varying, time-averaged force that results from the interaction of charged particles with rapidly oscillating electric and magnetic fields.

The ponderomotive force acting on a charged particle in an electromagnetic field can be derived from the Lorentz force. For an electron, the Lorentz force is given by:

\begin{equation}
    F_L = e \left( E + vB \right)
\end{equation}

where $e$ is the electron charge, $E$ is the electric field, $v$ is the electron velocity, and $B$ is the magnetic field.

When considering the interaction of an electron with a high-frequency electromagnetic field, it is useful to average the Lorentz force over several cycles of the field oscillations. This averaging yields the ponderomotive force, which can be expressed as:

\begin{equation}
    F_p = - \frac{e^2}{4 m \omega_0^2} \nabla |E|^2
\end{equation}

where $e$ is the electron charge, m is the electron mass, $\omega_0$ is the angular frequency of the electromagnetic field, and $|E|^2$ is the intensity of the electric field.

The ponderomotive force has several essential implications and applications, especially in the context of laser-plasma interactions:

Plasma acceleration: The ponderomotive force can drive electrons out of high-intensity field regions, creating density gradients and electric fields that can accelerate ions and other charged particles.

Generation of plasma waves and currents: The motion of electrons driven by the ponderomotive force can give rise to plasma waves and currents, which can generate magnetic fields and induce various plasma instabilities.

Formation of plasma channels: In high-intensity laser-plasma interactions, the ponderomotive force can expel electrons from the regions of highest intensity, forming low-density channels in the plasma. These channels can guide and focus laser pulses, leading to enhanced energy coupling and other interesting phenomena.

Laser-driven particle acceleration: The ponderomotive force plays a crucial role in several laser-driven particle acceleration schemes, such as laser wakefield acceleration (LWFA) and direct laser acceleration (DLA).

In summary, the ponderomotive force is a non-linear force experienced by charged particles in oscillating electromagnetic fields. It is an essential factor in various phenomena and applications related to laser-plasma interactions, including plasma acceleration, generation of magnetic fields, and laser-driven particle acceleration.