\section{Наброски}

Про движение электронов плазменного кластера твердотельной плотности под воздействием циркулярно-поляризованного импульса вроде бы всё понятно - есть в Andreev133.

В зависимости от массы ионов существуют параметры характерного размера мишени, интенсивности и длительности лазерного импульса, при которых влияние лазерного излучения на ионы уже не пренебрежимо мало. 

В результате взаимодействия с достаточно интенсивным импульсом формируется ударная волна ионной плотности внутри мишени, которая приводит к модуляции плотности в пространстве и времени, что влечет за собой локальное усиление магнитного поля в окрестности областей компрессии.

Вообще цилиндр лучше, потому что ориентация линий магнитного поля адекватнее - частицы будут тормозиться в направлении движения, магнитное поле и т.п. (сила Лоренца). Можно адекватнее манипулировать параметрами цилиндра