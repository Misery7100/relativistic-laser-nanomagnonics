\subsection{Генерация магнитного поля}

Вращающийся электрический диполь может генерировать направленное магнитное поле за счёт движения электрических зарядов. Простейшим примером вращающегося диполя является проволочная петля, по которой течет постоянный электрический ток.

Предположим, у нас есть вектор электрического дипольного момента $\vectbf{p}{}{}$, вращающийся с постоянной угловой скоростью $\omega$ вокруг декартовой оси $z$. Тогда временная зависимость электрического дипольного момента можно записать следующим образом:

\begin{equation}
    \vectbf{p}{}{}(t) = p \cdot \begin{pmatrix}\cos{\omega t} \\ \sin{\omega t} \\ 0\end{pmatrix},
\end{equation}

\noindent где $p$ --- амплитуда дипольного момента. Изменяющееся во времени электрическое поле диполя может быть определено при помощи следующего выражения:

\begin{equation}
    \vectbf{E}{}{} \left( \vectbf{r}{}{}, \: t \right) = \frac{1}{4 \pi \varepsilon_0} \frac{3 \left( \vectbf{p}{}{} \cdot \vectbf{r}{}{} \right) \vectbf{r}{}{} - r^2 \vectbf{p}{}{}}{r^5}.
\end{equation}

Магнитное поле, возникающее в результате вращения электрического диполя, может быть определено при помощи выражения закона Ампера:

\begin{equation}
    \nabla \times \vectbf{B}{}{} = \mu_0 \varepsilon_0 \frac{\partial \vectbf{E}{}{}}{\partial t}.
\end{equation}

Откуда немедленно следует, что магнитное поле направлено вдоль оси $z$, а его амплитуда зависит от амплитуды электрического дипольного момента и его угловой скорости.