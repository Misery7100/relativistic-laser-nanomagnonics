\section{Результаты моделирования}

\subsection{Одиночный кластер}

Было проведено 3D PIC моделирование взаимодействия Au$^{+30}$ кластера радиуса $R = 100$ nm с гауссовым импульсом c интенсивностью $I = 10^{20}\:\dots\: 10^{22}$ W/cm$^2$, длительностью $\tau = 10$ fs и длиной волны $\lambda = 1.064$ $\mu$m.

\img[img/new/Hx_by_intens]{Пиковая амплитуда магнитного поля $H_x$ в зависимости от интенсивности импульса при длительности $\tau = 10$ fs для одиночного кластера радиуса $R = 100$ nm.}{new_Hx_by_intens:image}{0.6\textwidth}

% Магнитное поле, генерируемое в результате взаимодействия с Au$^{+30}$ кластерами радиуса от 250 nm и выше при взаимодействии с циркулярно-поляризованным импульсом интенсивности порядка $10^{22}$ W/cm$^2$ 
Пиковое магнитное поле для Au$^{+30}$ кластеров при интенсивности циркулярно-поляризованного импульса порядка $10^{22}$ W/cm$^2$ и длительности $\tau = 10$ fs достигается при значении радиуса кластера $R = 200$ nm и составляет $H_x \approx 51$ GGs, что превышает интенсивность падающего циркулярно поляризованного импульса примерно на порядок. Полуширина магнитного поля на высоте в пространстве при достижении пикового значения составляет около 80 nm, распад магнитного поля адиабатически происходит за время около 10 fs~(\autoref{new_Hx_by_radius_by_t:image}).

Для значений радиуса выше 200 nm время жизни магнитного поля повышается, несмотря на то, что пиковое значение магнитного поля ниже в 2.5 раза, --- декремент затухания поля приблизительно в два раза ниже по сравнению с затуханием поля для кластера радиуса $R = 200$ nm.

\img[img/new/Hx_by_radius_by_t]{Амплитуда магнитного поля $H_x$ для различных значений радиуса кластера при длительности импульса $\tau = 10$ fs и интенсивности $I = 10^{22}$ W/cm$^2$.}{new_Hx_by_radius_by_t:image}{0.6\textwidth}

По сравнении с аналитической моделью, описанной в [DraftAAA], полученные результаты на максимум магнитного поля значительно отличаются~(\autoref{Hx_sim_vs_an:image}). В частности это связано с тем, что в описании этой модели не учитывается движение ионов, кроме изотропного теплового разлета кластера. 

Результаты моделирования и построения показывают, что в генерации магнитного поля косвенным образом участвуют и ионы, в частности благодаря сильным электростатическим силам между ионам и электронами --- любая модификация распределения ионов влияет на распределение электронов в пространстве, что в свою очередь оказывает влияние на магнитное поле, так как меняется общая конфигурация кольцевого тока электронов.

\img[img/new/Hx_sim_vs_an]{Пиковая амплитуда магнитного поля $H_x$ в зависимости от радиуса кластера при длительности импульса $\tau = 10$ fs и интенсивности $I = 10^{22}$ W/cm$^2$ в сравнении с аналитической моделью.}{Hx_sim_vs_an:image}{0.6\textwidth}

Первым фактором, модифицирующим ионную плотность, является пондеромоторная сила света. Для интенсивности порядка $10^{22}$ W/cm$^2$ пондеромоторная сила достаточно велика, чтобы деформировать начальное распределение плотности и продавить кластер в нормальной плоскости примерно на 10\%. Также того, что свет циркулярно поляризован, сжатие равномерно вдоль нормальной плоскости кластера. Вслед за уплотнением ионов на передней поверхности кластера происходит увеличение электронной плотности, что увеличивает кольцевой ток и генерируемое магнитное поле вслед за ним.

\begin{figure}[H]
    \subimgtwo[img/new/ni_sdf10_r200_2]{$t = 13.5$ fs.}{ni_200nm_plotly3d:a}{0.46\textwidth}
    \hfil
    \subimgtwo[img/new/ni_sdf12_r200_2]{$t = 16.5$ fs.}{ni_200nm_plotly3d:b}{0.46\textwidth}
    \\
    \subimgtwo[img/new/ni_sdf15_r200_2]{$t = 21$ fs.}{ni_200nm_plotly3d:c}{0.46\textwidth}
    \hfil
    \subimgtwo[img/new/ni_sdf16_r200_2]{$t = 22.5$ fs --- соответствует $H_{\max{}}$.}{ni_200nm_plotly3d:d}{0.46\textwidth}
    \caption{3D volume plot: ионная плотность $n_i$ (в единицах cm$^{-3}$) для кластера радиуса $R = 200$ nm в различные моменты времени. Для построения используется прозрачность.$\tau = 10$ fs, $I = 10^{22}$ W/cm$^2$.}\label{ni_200nm_plotly3d:image}
\end{figure}

Несмотря на то, что кластер испытывает тепловое расширение, большая часть ионов оказывается захвачена растущим магнитным полем и за время взаимодействия с импульсом постепенно стягивается к оси $x$, вдоль которой направлен падающий импульс, что по сути представляет собой пинч-эффект. 

Вкупе с пондеромоторной силой это приводит к модификации ионной (и электронной) плотности к конусовидной форме с последющей компрессией~(\autoref{ni_200nm_plotly3d:image}). 

% \img[img/new/yz_xy]{Пространственное распределение ионов Au$^{+30}$ для кластера с радиусом $R = 200$ nm в момент достижения пика магнитного поля.}{new_yz_xy:image}{0.85\textwidth}

\begin{figure}[H]
    \subimgtwo[img/new/ne_sdf10_r200_2]{$t = 13.5$ fs.}{ne_200nm_plotly3d:a}{0.46\textwidth}
    \hfil
    \subimgtwo[img/new/ne_sdf12_r200_2]{$t = 16.5$ fs.}{ne_200nm_plotly3d:b}{0.46\textwidth}
    \\
    \subimgtwo[img/new/ne_sdf15_r200_2]{$t = 21$ fs.}{ne_200nm_plotly3d:c}{0.46\textwidth}
    \hfil
    \subimgtwo[img/new/ne_sdf16_r200_2]{$t = 22.5$ fs --- соответствует $H_{\max{}}$.}{ne_200nm_plotly3d:d}{0.46\textwidth}
    \caption{3D volume plot: электронная плотность $n_e$ (в единицах cm$^{-3}$) для кластера радиуса $R = 200$ nm в различные моменты времени. Для построения используется прозрачность.$\tau = 10$ fs, $I = 10^{22}$ W/cm$^2$.}\label{ne_200nm_plotly3d:image}
\end{figure}

Построение электронной плотности показывает, что формирующийся электронный банч имеет конусовидную форму, а центральное распределение электронной плотности повторяет по форме ионное~(\autoref{ne_200nm_plotly3d:image}). Наибольший кольцевой ток из электронов образуется у основания конуса, что отвечает пику магнитного поля, причем приближенное изменение радиуса этого кольцевого тока хорошо коррелирует с генерируемым магнитным полем~(\autoref{Hx_200nm_1e22:image}, \autoref{Jx_200nm_1e22:image}). 

Для аналитической модели с неподвижными ионами минимальный радиус кольцевого тока задан радиусом кластера, что приводит к к расхождениям с численными результатами~(\autoref{Hx_sim_vs_an:image}). Для радиусов меньше некоторого предельного сила Ампера, обуславливающая пинч-эффект, оказывается достаточно высока, чтобы разрушить ток, что объясняет быстрый развал магнитного поля при $R < 200$ nm~(\autoref{new_Hx_by_radius_by_t:image}). При этом для радиусов выше этого предела канал существует некоторое время и магнитное поле стабилизируется.

\begin{figure}[H]
    \subimgtwo[img/new/bxsign_sdf10]{$t = 13.5$ fs.}{Hx_sign_200nm_1e22:a}{0.85\textwidth}
    \\
    \subimgtwo[img/new/bxsign_sdf14]{$t = 19.5$ fs.}{Hx_sign_200nm_1e22:b}{0.85\textwidth}
    \caption{Магнитное поле $H_x$ для кластера радиуса $R = 200$ nm.}\label{Hx_sign_200nm_1e22:image}
\end{figure}

\begin{figure}[H]
    \subimgtwo[img/new/bxsign_sdf16]{$t = 22.5$ fs.}{Hx_sign_200nm_1e22_2:b}{0.85\textwidth}
    \\
    \subimgtwo[img/new/bxsign_sdf18]{$t = 25.5$ fs.}{Hx_sign_200nm_1e22_2:b}{0.85\textwidth}
    \caption{Магнитное поле $H_x$ для кластера радиуса $R = 200$ nm.}\label{Hx_sign_200nm_1e22_2:image}
\end{figure}

\begin{figure}[H]
    \subimgtwo[img/new/jx_sdf10]{$t = 13.5$ fs.}{Jx_sign_200nm_1e22:a}{0.85\textwidth}
    \\
    \subimgtwo[img/new/jx_sdf14]{$t = 19.5$ fs.}{Jx_sign_200nm_1e22:b}{0.85\textwidth}
    \caption{Плотность тока $J_x$ для кластера радиуса $R = 200$ nm.}\label{Jx_sign_200nm_1e22:image}
\end{figure}

\begin{figure}[H]
    \subimgtwo[img/new/jx_sdf16]{$t = 22.5$ fs.}{Jx_sign_200nm_1e22_2:b}{0.85\textwidth}
    \\
    \subimgtwo[img/new/jx_sdf18]{$t = 25.5$ fs.}{Jx_sign_200nm_1e22_2:b}{0.85\textwidth}
    \caption{Плотность тока $J_x$ для кластера радиуса $R = 200$ nm.}\label{Jx_sign_200nm_1e22_2:image}
\end{figure}

\begin{figure}[H]
    \subimgtwo[img/new/Bx_xz_t24.0]{Диаметральная плоскость $xz$, $t = 22.5$ fs.}{Hx_200nm_1e22:a}{0.46\textwidth}
    \hfil
    \subimgtwo[img/new/Bx_yz_t24.0]{Диаметральная плоскость $yz$, $t = 22.5$ fs.}{Hx_200nm_1e22:b}{0.46\textwidth}
    \caption{Магнитное поле $H_x$ для кластера радиуса $R = 200$ nm в момент достижения пика магнитного поля.}\label{Hx_200nm_1e22:image}
\end{figure}

\begin{figure}[H]
    \subimgtwo[img/new/Jx_xz_t24.0]{Диаметральная плоскость $xz$, $t = 22.5$ fs.}{Jx_200nm_1e22:a}{0.46\textwidth}
    \hfil
    \subimgtwo[img/new/Jx_yz_t24.0]{Диаметральная плоскость $yz$, $t = 22.5$ fs.}{Jx_200nm_1e22:b}{0.46\textwidth}
    \caption{Ток $J_x$ для кластера радиуса $R = 200$ nm в момент достижения пика магнитного поля.}\label{Jx_200nm_1e22:image}
\end{figure}


\begin{figure}[H]
    \subimgtwo[img/new/x_px_R200]{Диаграмма $x$. $t = 22.5$ fs.}{pxpypz_200nm_1e22:e}{0.46\textwidth}
    \hfil
    \subimgtwo[img/new/y_py_R200]{Диаграмма $y$. $t = 22.5$ fs.}{pxpypz_200nm_1e22:f}{0.46\textwidth}
    \\
    \subimgtwo[img/new/z_pz_R200]{Диаграмма $z$. $t = 22.5$ fs.}{pxpypz_200nm_1e22:d}{0.46\textwidth}
    \caption{Фазовые диаграммы для кластера радиуса $R = 200$ nm в момент достижения пика магнитного поля.}\label{pxpypz_200nm_1e22:image}
\end{figure}

\img[img/new/phi]{Азимутальная фазовая диаграмма ионов Au$^{+30}$ для кластера с радиусом $R = 200$ nm в момент достижения пика магнитного поля и через 3 fs после.}{new_phi:image}{0.85\textwidth}