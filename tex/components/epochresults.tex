\section{Результаты моделирования}

\subsection{Одиночный кластер}

% Берем кластер радиуса 100 нм и кластер радиуса 50 нм. Интенсивность варьируется. Сначала графики с максимальным H (GGs) и релаксацией от времени (приблизительно). Дальше графики амбиполярной херни, фазовое пространство (мб), плотность и объяснение всей этой чечни. Чек аналитической зависимости. Сравнение с цилиндрами? (вряд ли имеет какой-то смысл)

Было проведено 3D PIC моделирование взаимодействия Au$^{+30}$ кластера радиуса $R = 100$ nm с гауссовым импульсом c интенсивностью $I = 10^{20}\:\dots\: 10^{22}$ W/cm$^2$, длительностью $\tau = 10$ fs и длиной волны $\lambda = 1.064$ $\mu$m.

В результате взаимодействия ионы и электроны кластера испытывают пондеромоторную силу, действующую со стороны вращающейся циркулярно поляризованной волны, которая в общем случае описывается как:

\begin{equation}
    F_p = - \frac{q^2}{4 m \omega^2} \nabla {|\vectbf{E}{}{}|}^2
\end{equation}

\noindent где $q$ --- заряд частицы, $m$ --- масса. Циркулярно поляризованное излучение как бы вворачивается в кластер, унося частицы за собой, а пондеромоторная сила сжимает мишень. За счёт круговой поляризации вектор пондеромоторной силы также вращается, что приводит к некоторой симметрии и равномерному сжатию мишени в плоскости, перпендикулярной плоскости падения импульса (\autoref{ni_100nm_1e22:image}). При этом в плоскости падения импульса сила неравномерна, что приводит к преобразованию мишени к форме, подобной цилиндру.

Помимо этого, этому также способствует задерживающее электроны возникающее в результате обратного эффекта Фарадея продольное магнитное поле. В результате общего действия продольного магнитного поля и пондеромоторной силы изначальная форма мишени преобразуется в некоторое кольцо или трубку с осью вдоль направления падения импульса, что видно по графикам ионной (\autoref{ni_100nm_1e22:image}), электронной плотности (\autoref{ne_100nm_1e22:image}), а также по распределению продольного электрического поля в момент времени $t = 20$ fs (\autoref{Ex_100nm_1e22:image}). Предполагается, что процесс сжатия будет более равномерный, если изначальная форма мишени будет цилиндрической --- так как в таком случае генерируемое магнитное поле будет выше и стабильнее за счет геометрии, которая обеспечивает более оптимальную организацию магнитных линий вдоль оси.

При интенсивности равной или ниже $10^{21}$ W/cm$^2$ действующая на кластер пондеромторная сила значительно ниже, что понижает пиковое значение магнитного поля, генерируемое в результате взаимодействия. Сравнение значений приведено на \autoref{Hx_vs_I:image}.

\begin{figure}[H]
    \subimgtwo[img/gold_cluster/1e21_100nm_10fs/ni_xz_t20.0]{Диаметральная плоскость $xz$, $t = 20$ fs.}{ni_100nm_1e21:e}{0.41\textwidth}
    \hfil
    \subimgtwo[img/gold_cluster/1e21_100nm_10fs/ni_yz_t20.0]{Диаметральная плоскость $yz$, $t = 20$ fs.}{ni_100nm_1e21:f}{0.41\textwidth}
    \caption{Ионная плотность $n_i$ для кластера радиуса $R = 100$ nm, интенсивность импульса $I = 10^{21}$ W/cm$^{2}$, длительность импульса $\tau = 10$~fs.}\label{ni_100nm_1e21:image}
\end{figure}

\begin{figure}[H]
    \subimgtwo[img/gold_cluster/1e21_100nm_10fs/ne_xz_t20.0]{Диаметральная плоскость $xz$, $t = 20$ fs.}{ne_100nm_1e21:e}{0.41\textwidth}
    \hfil
    \subimgtwo[img/gold_cluster/1e21_100nm_10fs/ne_yz_t20.0]{Диаметральная плоскость $yz$, $t = 20$ fs.}{ne_100nm_1e21:f}{0.41\textwidth}
    \caption{Электронная плотность $n_e$ для кластера радиуса $R = 100$ nm, интенсивность импульса $I = 10^{21}$ W/cm$^{2}$, длительность импульса $\tau = 10$~fs.}\label{ne_100nm_1e21:image}
\end{figure}

\begin{figure}[H]
    \subimgtwo[img/gold_cluster/1e20_100nm_10fs/ni_xz_t20.0]{Диаметральная плоскость $xz$, $t = 20$ fs.}{ni_100nm_1e20:e}{0.41\textwidth}
    \hfil
    \subimgtwo[img/gold_cluster/1e20_100nm_10fs/ni_yz_t20.0]{Диаметральная плоскость $yz$, $t = 20$ fs.}{ni_100nm_1e20:f}{0.41\textwidth}
    \caption{Ионная плотность $n_i$ для кластера радиуса $R = 100$ nm, интенсивность импульса $I = 10^{20}$ W/cm$^{2}$, длительность импульса $\tau = 10$~fs.}\label{ni_100nm_1e20:image}
\end{figure}

\begin{figure}[H]
    \subimgtwo[img/gold_cluster/1e20_100nm_10fs/ne_xz_t20.0]{Диаметральная плоскость $xz$, $t = 20$ fs.}{ne_100nm_1e20:e}{0.41\textwidth}
    \hfil
    \subimgtwo[img/gold_cluster/1e20_100nm_10fs/ne_yz_t20.0]{Диаметральная плоскость $yz$, $t = 20$ fs.}{ne_100nm_1e20:f}{0.41\textwidth}
    \caption{Электронная плотность $n_e$ для кластера радиуса $R = 100$ nm, интенсивность импульса $I = 10^{20}$ W/cm$^{2}$, длительность импульса $\tau = 10$~fs.}\label{ne_100nm_1e20:image}
\end{figure}

\begin{figure}[H]
    \subimgtwo[img/gold_cluster/1e22_100nm_10fs/Bx_xz_t10.0]{Диаметральная плоскость $xz$, $t = 10$ fs.}{Hx_100nm_1e22:a}{0.41\textwidth}
    \hfil
    \subimgtwo[img/gold_cluster/1e22_100nm_10fs/Bx_yz_t10.0]{Диаметральная плоскость $yz$, $t = 10$ fs.}{Hx_100nm_1e22:b}{0.41\textwidth}
    \subimgtwo[img/gold_cluster/1e22_100nm_10fs/Bx_xz_t15.0]{Диаметральная плоскость $xz$, $t = 15$ fs.}{Hx_100nm_1e22:c}{0.41\textwidth}
    \hfil
    \subimgtwo[img/gold_cluster/1e22_100nm_10fs/Bx_yz_t15.0]{Диаметральная плоскость $yz$, $t = 15$ fs.}{Hx_100nm_1e22:d}{0.41\textwidth}
    \subimgtwo[img/gold_cluster/1e22_100nm_10fs/Bx_xz_t20.0]{Диаметральная плоскость $xz$, $t = 20$ fs.}{Hx_100nm_1e22:e}{0.41\textwidth}
    \hfil
    \subimgtwo[img/gold_cluster/1e22_100nm_10fs/Bx_yz_t20.0]{Диаметральная плоскость $yz$, $t = 20$ fs.}{Hx_100nm_1e22:f}{0.41\textwidth}
    \caption{Продольная компонента магнитного поля $H_x$ для кластера радиуса $R = 100$ nm, интенсивность импульса $I = 10^{22}$ W/cm$^{2}$, длительность импульса $\tau = 10$~fs.}\label{Hx_100nm_1e22:image}
\end{figure}

\begin{figure}[H]
    \subimgtwo[img/gold_cluster/1e22_100nm_10fs/Ex_xz_t10.0]{Диаметральная плоскость $xz$, $t = 10$ fs.}{Ex_100nm_1e22:a}{0.41\textwidth}
    \hfil
    \subimgtwo[img/gold_cluster/1e22_100nm_10fs/Ex_yz_t10.0]{Диаметральная плоскость $yz$, $t = 10$ fs.}{Ex_100nm_1e22:b}{0.41\textwidth}
    \subimgtwo[img/gold_cluster/1e22_100nm_10fs/Ex_xz_t15.0]{Диаметральная плоскость $xz$, $t = 15$ fs.}{Ex_100nm_1e22:c}{0.41\textwidth}
    \hfil
    \subimgtwo[img/gold_cluster/1e22_100nm_10fs/Ex_yz_t15.0]{Диаметральная плоскость $yz$, $t = 15$ fs.}{Ex_100nm_1e22:d}{0.41\textwidth}
    \subimgtwo[img/gold_cluster/1e22_100nm_10fs/Ex_xz_t20.0]{Диаметральная плоскость $xz$, $t = 20$ fs.}{Ex_100nm_1e22:e}{0.41\textwidth}
    \hfil
    \subimgtwo[img/gold_cluster/1e22_100nm_10fs/Ex_yz_t20.0]{Диаметральная плоскость $yz$, $t = 20$ fs.}{Ex_100nm_1e22:f}{0.41\textwidth}
    \caption{Продольная компонента электрического поля $E_x$ для кластера радиуса $R = 100$ nm, интенсивность импульса $I = 10^{22}$ W/cm$^{2}$, длительность импульса $\tau = 10$~fs.}\label{Ex_100nm_1e22:image}
\end{figure}

\begin{figure}[H]
    \subimgtwo[img/gold_cluster/1e22_100nm_10fs/ni_xz_t10.0]{Диаметральная плоскость $xz$, $t = 10$ fs.}{ni_100nm_1e22:a}{0.41\textwidth}
    \hfil
    \subimgtwo[img/gold_cluster/1e22_100nm_10fs/ni_yz_t10.0]{Диаметральная плоскость $yz$, $t = 10$ fs.}{ni_100nm_1e22:b}{0.41\textwidth}
    \subimgtwo[img/gold_cluster/1e22_100nm_10fs/ni_xz_t15.0]{Диаметральная плоскость $xz$, $t = 15$ fs.}{ni_100nm_1e22:c}{0.41\textwidth}
    \hfil
    \subimgtwo[img/gold_cluster/1e22_100nm_10fs/ni_yz_t15.0]{Диаметральная плоскость $yz$, $t = 15$ fs.}{ni_100nm_1e22:d}{0.41\textwidth}
    \subimgtwo[img/gold_cluster/1e22_100nm_10fs/ni_xz_t20.0]{Диаметральная плоскость $xz$, $t = 20$ fs.}{ni_100nm_1e22:e}{0.41\textwidth}
    \hfil
    \subimgtwo[img/gold_cluster/1e22_100nm_10fs/ni_yz_t20.0]{Диаметральная плоскость $yz$, $t = 20$ fs.}{ni_100nm_1e22:f}{0.41\textwidth}
    \caption{Ионная плотность $n_i$ для кластера радиуса $R = 100$ nm, интенсивность импульса $I = 10^{22}$ W/cm$^{2}$, длительность импульса $\tau = 10$~fs.}\label{ni_100nm_1e22:image}
\end{figure}

\begin{figure}[H]
    \subimgtwo[img/gold_cluster/1e22_100nm_10fs/ne_xz_t10.0]{Диаметральная плоскость $xz$, $t = 10$ fs.}{ne_100nm_1e22:a}{0.41\textwidth}
    \hfil
    \subimgtwo[img/gold_cluster/1e22_100nm_10fs/ne_yz_t10.0]{Диаметральная плоскость $yz$, $t = 10$ fs.}{ne_100nm_1e22:b}{0.41\textwidth}
    \subimgtwo[img/gold_cluster/1e22_100nm_10fs/ne_xz_t15.0]{Диаметральная плоскость $xz$, $t = 15$ fs.}{ne_100nm_1e22:c}{0.41\textwidth}
    \hfil
    \subimgtwo[img/gold_cluster/1e22_100nm_10fs/ne_yz_t15.0]{Диаметральная плоскость $yz$, $t = 15$ fs.}{ne_100nm_1e22:d}{0.41\textwidth}
    \subimgtwo[img/gold_cluster/1e22_100nm_10fs/ne_xz_t20.0]{Диаметральная плоскость $xz$, $t = 20$ fs.}{ne_100nm_1e22:e}{0.41\textwidth}
    \hfil
    \subimgtwo[img/gold_cluster/1e22_100nm_10fs/ne_yz_t20.0]{Диаметральная плоскость $yz$, $t = 20$ fs.}{ne_100nm_1e22:f}{0.41\textwidth}
    \caption{Электронная плотность $n_e$ для кластера радиуса $R = 100$ nm, интенсивность импульса $I = 10^{22}$ W/cm$^{2}$, длительность импульса $\tau = 10$~fs.}\label{ne_100nm_1e22:image}
\end{figure}

\img[img/gold_cluster/Hx_by_intens]{Максимальная амплитуда (по модулю) продольной компоненты магнитного поля $H_x$ в зависимости от интенсивности импульса при длительности импульса $\tau = 10$~fs, радиусе кластера $R = 100$~nm.}{Hx_vs_I:image}{0.6\textwidth}